\begin{titlepage}
\pagestyle{empty}
%\begin{latexonly}
\begin{center}
%\section*{ }
\begin{tabular}{lr|l}
  &\textsf{NCAR/TN--556+STR}&\hspace{0.5cm}{       }\\
  &\textsf{\textbf{NCAR TECHNICAL NOTE}}&\\ \hline
  &March 2019&\\[1cm]
\multicolumn{2}{l|}
{\LARGE \textsf{\textbf{A Description of the \hphantom{Advanced Research WRF }}}}
&\\ [5pt]

\multicolumn{2}{l|}
{\LARGE \textsf{\textbf{Advanced Research WRF Model Version 4}}}
&\\[1cm]

\normalsize
William C. Skamarock&&\\
Joseph B. Klemp&&\\
Jimy Dudhia&&\\
David O. Gill&&\\
Zhiquan Liu&&\\
Judith Berner&&\\
Wei Wang&&\\
Jordan G. Powers&&\\
Michael G. Duda&&\\
Dale M. Barker&&\\
Xiang-Yu Huang&&\\[11cm]
&&\\[-1cm]
%&&\\%[15cm]

\multicolumn{2}{r|}{Mesoscale and Microscale Meteorology Laboratory}&\\ \hline
\multicolumn{2}{r|}{National Center for Atmospheric Research}&\\
\multicolumn{2}{r|}{Boulder, Colorado, USA}&\\

\end{tabular}
\end{center}
%\end{latexonly}

\newpage
\vspace{1cm}
\begin{center}
\textbf{NCAR TECHNICAL NOTES}
\end{center}
\vspace{1.5cm}
The Technical Note series provides an outlet for a variety of NCAR
manuscripts that contribute in specialized ways to the body of
scientific knowledge but which are not suitable for journal,
monograph, or book publication.  Reports in this series are issued by
the NCAR Scientific Divisions; copies may be obtained on request from
the Publications Office of NCAR.  Designation symbols for the series
include:\\[1cm]
\begin{tabbing}
\hspace{3cm}\=\hspace{1.6cm}\=\kill
\\
\> EDD: \>\emph{Engineering, Design, or Development Reports}\\
\> \> Equipment descriptions, test results, instrumentation,\\
\> \> and operating and maintenance manuals.\\[2ex]
\> IA: \>\emph{Instructional Aids}\\
\> \> Instruction manuals, bibliographies, film supplements,\\
\> \> and other research or instructional aids.\\[2ex]
\> PPR: \>\emph{Program Progress Reports}\\
\> \> Field program reports, interim and working reports,\\
\> \> survey reports, and plans for experiments.\\[2ex]
\> PROC: \>\emph{Proceedings}\\
\> \> Documentation of symposia, colloquia, conferences, workshops,\\
\> \> and lectures.  (Distribution may be limited to attendees.)\\[2ex]
\> STR: \>\emph{Scientific and Technical Reports}\\
\> \> Data compilations, theoretical and numerical\\
\> \> investigations, and experimental results.\\[2cm]
\end{tabbing}
\emph{The National Center for Atmospheric Research (NCAR) is 
operated by the University Corporation for Atmospheric Research
(UCAR) and is sponsored by the National Science Foundation.
Any opinions, findings, conclusions, or recommendations expressed
in this publication are those of the authors and do not 
necessarily reflect the views of the National Science Foundation.}

\newpage
\vspace*{2cm}
\vskip 24pt
\begin{center}
\textbf{Foreword}
\end{center}
\vspace{1cm}
This Tech Note describes the dynamics, solver, physics, and data assimilation aspects of the Advanced
Research WRF model Version 4. This particular version of the Tech Note covers ARW releases up to Version 4.1.
This document will be updated as new releases become available and new features are added to the model.

\vfill
\pagebreak
\hphantom{This page is left blank}

\rmfamily
\end{titlepage}
\pagenumbering{roman}
\setcounter{page}{1}
