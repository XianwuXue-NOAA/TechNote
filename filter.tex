\chapter{Turbulent Mixing and Model Filters}
\label{filter_chap}

A number of formulations for turbulent mixing and filtering are
available in the ARW solver.  Some of these filters are used for
numerical reasons.  For example, divergence damping is used to filter
acoustic modes from the solution and polar filtering is used to reduce 
the timestep restriction arising from the converging gridlines of the 
latitude-longitude grid.  Other filters are meant to represent
sub-grid turbulence processes that cannot be resolved on the chosen
grid.  These filters remove energy from the solution and are formulated
in part on turbulence theory and observations, or represent energy sink
terms in some approximation to the Euler equation.  In this section, we
begin by outlining the formulation and discretization of turbulent
mixing processes in the ARW solver commonly associated with sub-gridscale
turbulence as parameterized in cloud-scale models--- the second-order
horizontal and vertical mixing.  In large-scale models and most NWP
models, vertical mixing is parameterized within the planetary boundary
layer (PBL) physics.  Vertical mixing parameterized within the PBL
physics is described later in Chapter \ref{physics_chap}.  Here we note
that, when a PBL parameterization is used, all other vertical mixing is
disabled.  Following the outline of turbulent mixing parameterizations
in this chapter, other numerical filters available in the ARW solver are
described.


\section{Latitude-Longitude Global Grid and Polar Filtering}
\label{polar_filter_section}

Polar filtering is used to reduce the timestep restriction associated
with the gridlines that converge as they approach the poles of the
latitude-longitude grid; the converging gridlines reduce the
longitudinal gridlength, and thus the stable timestep for transport,
discussed in Section \ref{rk3_timestep_constraint}, and for acoustic
waves, discussed in Section \ref{acoustic_step_constraint}, will be reduced.

Polar filtering a given variable is accomplished by first applying a 1D
Fourier transform to the variable on a constant computational-grid
latitude circle (the forward transform), where the field is periodic.
The Fourier coefficients with wavenumbers above a prescribed threshold
are truncated, after which a transformation back to physical space is
applied (the backward transform), completing the filter step.

A filter application can be written as
%
\begin{equation}
\hat\phi(k)_{filtered} = a(k)\,\hat\phi(k), \,\,\,\,\, \hbox{for all }k,
\notag
\end{equation}
%
where $\hat\phi(k)$ and $\hat\phi(k)_{filtered}$ are the Fourier
coefficients for the generic variable $\phi$ before and after filtering,
and $a(k)$ are the filter coefficients defined as a function of the
dimensionless wavenumber $k$.  The ARW polar filter coefficients $a(k)$ 
for the Fourier amplitudes are
%
\begin{equation}
a(k) = \hbox{min}\left[1.,\hbox{max}\left(0.,
\left({\cos{\psi} \over \cos{\psi_o}}\right)^2  {1 \over \sin^2({\pi k/n)}}
\right)\right]
\notag
%\label{polar-filter}
\end{equation}
%
where $\psi$ is the latitude on the computational grid, $\psi_o$ is the
latitude above which the polar filter is applied (the filter is applied
for $ |\psi| > \psi_o$), and $n$ is the number of grid points in the latitude
circle.  For even $n$, the grid
admits wavenumber $k=0$ and dimensionless wavenumbers $\pm k$ for
$k=1\to n/2 - 1$, and the $2\Delta$ wave $k=n/2$.  For odd $n$, the grid
admits wavenumber $k=0$, and dimensionless wavenumbers $\pm k$ for
$k=1\to (n-1)/2 $. 

Polar filter applications within the split-explicit RK3 time integration
scheme are given in Figure \ref{time_integration_figure}.
Within the acoustic steps,
the filter is applied to
${U''}^{\tau+\Delta\tau}$ and ${V''}^{\tau+\Delta\tau}$ immediately
after the horizontal momentum is advanced, to $p_c^{\tau+\Delta\tau}$
and ${\Theta_m''}^{\tau+\Delta\tau}$ immediately after the column mass and
the potential temperature are advanced, and to ${W''}^{\tau+\Delta\tau}$
and ${\phi''}^{\tau+\Delta\tau}$ after the vertically implicit part of
the acoustic step is complete.  The polar filter is also applied to the
scalars after they are advanced every RK3 sub-step
\eqref{rk3a}-\eqref{rk3c} and after the microphysics step.

The prognostic variables are coupled with the column mass $\mu$ when
filtered, and in this way mass is conserved.  An exception to this is
the geopotential $\phi$ which is filtered without being coupled.  The
Fourier filtering is conservative, but it is not monotonic or positive
definite.  As noted in the ARW stability discussion in Section
\ref{global_considerations}, timestep stability should be based on the longitudinal
gridlength where the polar filters are activated, $(\Delta
x/m_x)|_{\psi_o}$.  The positive definite transport option, presented in
Section \ref{positive-definite-transport}, will not necessarily produce
positive-definite results because of the converging gridlines and should
not be used.

\section{Explicit Spatial Diffusion}

The ARW solver has three formulations for spatial dissipation--- diffusion
along coordinate surfaces, diffusion in physical $(x,y,z)$ space, and 
a sixth-order diffusion applied on horizontal coordinate surfaces.  In
the following sections we present the diffusion operators for the the
first two
formulations, followed by the four separate formulations that can be
used to compute the eddy viscosities and a description of
the prognostic turbulent kinetic energy (TKE) equation used in one set
of these formulations.  The sixth order spatial filter is described at
the end of this section. In these formulations, the horizontal ($K_h$) and
vertical ($K_v$) eddy viscosities are defined at scalar points on the 
staggered model grid. 

\subsection{Horizontal and Vertical Diffusion on Coordinate Surfaces}

For any model variable, horizontal and vertical second order spatial
filtering on model coordinate surfaces is considered part of the RHS
terms in the continuous equations 
\eqref{u-mom-pert} -- \eqref{scalar-pert}
and can be expressed as follows for
a model variable $a$:
%
\begin{equation}
\partial_t  (\mu_d a)  = ... \,\,
+ \mu_d \bigl[ m_x \partial_x(m_x K_h \partial_x  a) 
           + m_y \partial_y(m_y K_h \partial_y  a)\bigr]
+ g^2(\mu_d\alpha)^{-1} \partial_\eta(K_v \alpha^{-1} \partial_\eta  a).
\label{dissipation_1}
\end{equation}
%
For the horizontal and vertical momentum equations, 
\eqref{dissipation_1} is spatially discretized as
%
\begin{align}
\partial_t U &= ... \,\, +
{\overline{\mu_d}}^x {\overline{m_x}}^x
\bigl[ \delta_x(m_x K_h \delta_x u) 
+ \delta_y({\overline{m_y}}^{xy} {\overline{K_h}}^{xy} \delta_y u)  \bigr]
+ m_y^{-1} g^2({\overline{\mu_d}}^x {\overline\alpha}^x)^{-1} 
%% \delta_\eta(K_v ({\overline \alpha}^{x\eta})^{-1} \delta_\eta u) 
\delta_\eta({\overline K_v}^{x\eta} ({\overline \alpha}^{x\eta})^{-1} \delta_\eta u) 
\notag \\
%
\partial_t V &= ... \,\, +
{\overline{\mu_d}}^y {\overline{m_y}}^y
\bigl[ 
\delta_x({\overline{m_x}}^{xy} {\overline K_h}^{xy} \delta_x v)  
+\delta_y(m_y K_h \delta_y v) 
\bigr]
+ m_x^{-1} g^2({\overline{\mu_d}}^y {\overline\alpha}^y)^{-1} 
%% \delta_\eta(K_v ({\overline \alpha}^{y\eta})^{-1} \delta_\eta v) 
\delta_\eta({\overline K_v}^{y\eta} ({\overline \alpha}^{y\eta})^{-1} \delta_\eta v) 
\notag \\
%
\partial_t W &= ... \,\, +
\mu_d m_x
\bigl[ 
\delta_x({\overline m_x}^{x} {\overline K_h}^{x\eta} \delta_x w)  
+\delta_y({\overline m_y}^{y} {\overline K_h}^{y\eta} \delta_y w)  
\bigr]
+ m_y^{-1} g^2({\mu_d} {\overline\alpha}^\eta)^{-1} 
\delta_\eta(K_v \alpha^{-1} \delta_\eta w).
\notag
\end{align}
%
\noindent
The spatial discretization for a scalar $ q$, defined at the mass
points, is
%
\begin{equation}
\partial_t (\mu_d q) = ... \,\, +
\mu_d m_x m_y
\bigl[ 
\delta_x({\overline m_x}^{x} P_r^{-1} {\overline K_h}^{x} \delta_x  q)  
+\delta_y({\overline m_y}^{y} P_r^{-1} {\overline K_h}^{y} \delta_y  q)  
\bigr]
+ g^2({\mu_d} {\alpha})^{-1} 
%% \delta_\eta(K_v \alpha^{-1} \delta_\eta  q). 
\delta_\eta({\overline K_v}^\eta ({\overline \alpha}^\eta)^{-1} \delta_\eta  q). 
\notag
\end{equation}
%
\noindent
In the current ARW formulation for mixing on coordinate surfaces, the
horizontal eddy viscosity $K_h$ is allowed to vary in space, whereas the
vertical eddy viscosity does not vary in space; hence there is no need
for any spatial averaging of $K_v$.  Additionally, note that the
horizontal eddy viscosity $K_h$ is multiplied by the inverse turbulent
Prandtl number $P_r^{-1}$ for horizontal scalar mixing.

\subsection{Horizontal and Vertical Diffusion in Physical Space}

\subsubsection{Coordinate Metrics}

We use the geometric height coordinate in this physical space
formulation.
The coordinate metrics are computed using the prognostic geopotential 
in the ARW solver.  At the beginning of each Runge-Kutta time step,
the coordinate metrics are evaluated as part of the overall 
algorithm.  The definitions of the metrics are
%
\begin{equation}
z_x = g^{-1} \delta_x \phi \,\,\, \hbox{and} \,\,\, 
z_y = g^{-1} \delta_y \phi.
\notag
\end{equation}
%
\noindent
These metric terms are defined on $w$ levels, and
 $(z_x, z_y)$ are 
horizontally  coincident with ($u$,$v$) points.
Additionally, the vertical  diffusion terms are evaluated
directly in terms of the geometric height, avoiding the need
for metric terms in the vertical.

\subsubsection{Continuous Equations}

The continuous equations for evaluating diffusion in physical space,
using the velocity stress tensor, are as follows for
horizontal and vertical momentum:
%
\begin{align}
\partial_t U &= ...\,\,
- m_x \bigl[ \partial_x \tau_{11} + \partial_y \tau_{12} 
- z_x \partial_z  \tau_{11} - z_y \partial_z \tau_{12} \bigr]
- \partial_z \tau_{13}
\label{u-disp-phys} \\
%
\partial_t V &= ...\,\,
- m_y \bigl[ \partial_x \tau_{12} + \partial_y \tau_{22} 
- z_x \partial_z \tau_{12} - z_y \partial_z \tau_{22} \bigr]
- \partial_z \tau_{23}
\label{v-disp-phys} \\
%
\partial_t W &= ...\,\,
- m_y \bigl[ \partial_x \tau_{13} + \partial_y \tau_{23} 
- z_x \partial_z \tau_{13} - z_y \partial_z \tau_{23} \bigr]
- \partial_z \tau_{33}.
\label{w-disp-phys}
\end{align}
%
The stress tensor $\tau$ can be written as follows:
%
\begin{align}
\tau_{11} &= -\mu_d K_h D_{11} 
\notag \\
%
\tau_{22} &= -\mu_d K_h D_{22} 
\notag \\
%
\tau_{33} &= -\mu_d K_v D_{33} 
\notag \\
%
\tau_{12} &= -\mu_d K_h D_{12} 
\notag \\
%
\tau_{13} &= -\mu_d K_v D_{13} 
\notag \\
%
\tau_{23} &= -\mu_d K_v D_{23}.
\notag 
%
\end{align}
%
\noindent
Symmetry sets the remaining tensor values;
$\tau_{21} = \tau_{12}$, $\tau_{31} = \tau_{13}$, and $\tau_{32} = \tau_{23}$.
The stress tensor $\tau$ is calculated from the deformation tensor
$D$.  The continuous deformation tensor is defined as
%
\begin{align}
D_{11} &= 2 \, m_x m_y \bigl[ \partial_x (m_y^{-1}u) - z_x \partial_z
(m_y^{-1}u) \bigl] 
\notag \\
%
D_{22} &= 2 \, m_x m_y \bigl[ \partial_y (m_x^{-1}v) - z_y \partial_z
(m_x^{-1}v) \bigl] 
\notag \\
%
D_{33} &= 2 \, \partial_z w
\notag \\
%
D_{12} &= m_x m_y \bigl[ 
\partial_y (m_y^{-1}u) - z_y \partial_z(m_y^{-1}u)
+ \partial_x (m_x^{-1}v) - z_x \partial_z (m_x^{-1}v)
\bigl] 
\notag \\
%
D_{13} &= m_x m_y \bigl[ 
\partial_x (m_y^{-1}w) - z_x \partial_z(m_y^{-1}w)
\bigl] 
+ \partial_z (u)
\notag \\
%
D_{23} &= m_x m_y \bigl[ 
\partial_y (m_y^{-1}w) - z_y \partial_z(m_y^{-1}w)
\bigl] 
+ \partial_z (v).
\notag 
%
\end{align}
%
\noindent
The deformation tensor is symmetric, hence 
$D_{21} = D_{12}$, $D_{31} = D_{13}$, and $D_{32} = D_{23}$.

The diffusion formulation for scalars is
%
\begin{align}
\partial_t (\mu_d q) = ...\,\, + \bigl[
& \, m_x \bigl(\partial_x - \partial_z z_x \bigr )
%% \bigl( \mu_d m_x K (\partial_x - z_x \partial_z ) \bigr) 
\bigl( \mu_d m_x K_h (\partial_x - z_x \partial_z ) \bigr) 
+ 
\notag \\
& \, m_y \bigl(\partial_y - \partial_z z_y \bigr )
%% \bigl( \mu_d m_y K (\partial_y - z_y \partial_z ) \bigr) 
\bigl( \mu_d m_y K_h (\partial_y - z_y \partial_z ) \bigr) 
%% +  \partial_z \mu_d K \partial_z 
+  \partial_z \mu_d K_v \partial_z 
\bigr]  q.
\label{scalar-disp-phys}
\end{align}

\subsubsection{Spatial Discretization}

Using the definition of the stress tensor, the spatial discretization of the
ARW physical-space diffusion operators for the horizontal and vertical
momentum equations
\eqref{u-disp-phys} - \eqref{w-disp-phys} are
%
\begin{align}
\partial_t U &= ...\,\,
- {\overline{m_x}}^x \bigl[ \delta_x \tau_{11} + \delta_y \tau_{12} 
- \overline{z_x}^{\eta} \delta_z 
{\overline{\tau_{11}}}^{x\eta} - {\overline {z_y}}^{xy\eta} \delta_z 
{\overline{\tau_{12}}}^{y\eta} \bigr]
-\delta_z \tau_{13}
\notag \\
%
\partial_t V &= ...\,\,
- {\overline{m_y}}^y \bigl[ \delta_y \tau_{22} + \delta_x \tau_{12} 
- \overline{z_y}^{\eta} \delta_z 
{\overline{\tau_{22}}}^{y\eta} - {\overline {z_x}}^{xy\eta} \delta_z
{\overline{\tau_{12}}}^{x\eta} \bigr]
-\delta_z \tau_{23}
\notag \\
%
\partial_t W &= ...\,\,
- m_x \bigl[ \delta_x \tau_{13} + \delta_y \tau_{23} 
- \overline {z_x}^{x} \delta_z 
{\overline{\tau_{13}}}^{x\eta} 
-{\overline {z_y}}^{y} \delta_z{\overline{\tau_{23}}}^{y\eta} 
\bigr]
- \delta_z \tau_{33}.
\notag
\end{align}
%
\noindent
The discrete forms of the stress tensor and deformation tensor are
\noindent
%
\begin{align}
\tau_{11} &= -\mu_d K_h D_{11} 
\notag \\
%
\tau_{22} &= -\mu_d K_h D_{22} 
\notag \\
%
\tau_{33} &= -\mu_d K_v D_{33} 
\notag \\
%
\tau_{12} &= -\overline {\mu_d}^{xy} 
\overline {K_h}^{xy} D_{12} 
\notag \\
%
\tau_{13} &= -\overline {\mu_d}^{x} 
\overline {K_v}^{x\eta} D_{13} 
\notag \\
%
\tau_{23} &= -\overline {\mu_d}^{y} 
\overline {K_v}^{y\eta} D_{23},
\notag
%
\end{align}
%
\noindent
and
%
\begin{align}
D_{11} &= 2 \, m_x m_y \bigl[ \delta_x ({\overline {m_y^{-1}}}^x u) 
- {\overline{z_x}}^{x\eta} \delta_z ({\overline {m_y^{-1}}}^x u) \bigl] 
\notag \\
%
D_{22} &= 2 \, m_x m_y \bigl[ \delta_y ({\overline {m_x^{-1}}}^y v) 
- {\overline{z_y}}^{x\eta} \delta_z ({\overline {m_x^{-1}}}^y v) \bigl] 
\notag \\
%
D_{33} &= 2 \, \delta_z w
\notag \\
%
D_{12} &= (\overline{m_x m_y}^{xy}) \biggl[ 
\delta_y ({\overline {m_y^{-1}}}^x u) 
- {\overline{z_y}}^{x\eta} 
\delta_z {\overline{({\overline {m_y^{-1}}}^x u)}^{y\eta}} 
+ \delta_x ({\overline {m_x^{-1}}}^y v) 
- {\overline{z_x}}^{y\eta}
\delta_z {\overline{({\overline {m_x^{-1}}}^y v)}^{x\eta}} 
\biggl] 
\notag \\
%
D_{13} &= m_x m_y \biggl[ 
\delta_x (m_y^{-1} w) 
- z_x \delta_z \overline{(m^{-1} w)}^{x\eta}
\biggl] 
+ \delta_z u
\notag \\
%
D_{23} &= m_x m_y \biggl[ 
\delta_y (m_y^{-1} w) 
- z_y \delta_z \overline{(m_y^{-1} w)}^{y\eta}
\biggl] 
+ \delta_z v.
\notag
%
\end{align}

\noindent
The spatial discretization for the scalar diffusion
\eqref{scalar-disp-phys} is
%
\begin{align}
\partial_t (\mu_d q) = ...\,\, & 
+ m_x \bigl[
\delta_x \bigl({\overline \mu_d}^x H_1( q) \bigr)
- \mu_d \overline{z_x}^{x\eta}\delta_z 
\bigl( {\overline{H_1( q)}}^{x\eta} \bigr)
\bigr] 
\notag \\
& + m_y \bigl[
\delta_y \bigl({\overline \mu_d}^y H_2( q) \bigr)
- \mu_d \overline{z_y}^{y\eta}\delta_z 
\bigl( {\overline{H_2( q)}}^{y\eta} \bigr)
\bigr] 
\notag \\
& + \mu_d \delta_z \bigl({\overline{K_v}}^\eta \delta_z  q \bigr),
\notag
\end{align}
%
\noindent
where
\begin{align}
H_1( q) &= \overline{m_x}^x {\overline{K_h}}^x \bigl(
\delta_x q - z_x \delta_z( {\overline{ q}}^{x\eta} )\bigr),
\notag \\
H_2( q) &= \overline{m_y}^y {\overline{K_h}}^y \bigl(
\delta_y q - z_y \delta_z( {\overline{ q}}^{y\eta} )\bigr).
\notag
\end{align}

\subsection{Computation of the Eddy Viscosities}
\label {eddy_section}

There are four options for determining the eddy viscosities $K_h$ and
$K_v$ in the ARW solver.

\subsubsection{External specification of $K_h$ and $K_v$}

Constant values for $K_h$ and $K_v$ can be input in the ARW namelist.

\subsubsection{$K_h$ determined from the horizontal deformation}

The horizontal eddy viscosity $K_h$ can be determined from the
horizontal deformation using a Smagorinsky first-order closure
approach.  In this formulation, the eddy viscosity is defined 
and discretized as
%
\begin{equation}
K_h = C_s^2 \, l^2 \biggl[ 
0.25(D_{11}-D_{22})^2 + {\overline{D_{12}^2}}^{xy}
\biggr]^{1 \over 2}.
\notag
\end{equation}
%
\noindent
The deformation tensor components have been defined in the previous
section.  The length scale $l = (\Delta x \Delta y)^{1/2}$ and $C_s$ is
a constant with a typical value $C_s = 0.25$.  For scalar mixing, the
eddy viscosity is divided by the turbulent Prandtl number $P_r$ that
typically has a value of 1/3 \citep{deardorff72}.  This option is most
often used with a planetary boundary layer scheme that independently
handles the vertical mixing.

\subsubsection{3D Smagorinsky Closure}

The horizontal and vertical eddy viscosities can be determined using a 
3D Smagorinsky turbulence closure.  This closure specifies the
eddy viscosities as

\begin{equation}
K_{h,v} = C_s^2 \, l_{h,v}^2 \,\hbox{max}
\biggl[ 0., \bigl( D^2 - P_r^{-1} N^2 \bigr)^{1/2}\biggr],
\label{k_calc}
\end{equation}
%
\noindent
where 
%
\begin{equation}
D^2 = 
{1 \over 2} \biggl[
D^2_{11} +
D^2_{22} +
D^2_{33} \biggr] +
\bigl({{\overline{D_{12}}}^{xy}}\bigr)^2 +
\bigl({{\overline{D_{13}}}^{x\eta}}\bigr)^2 +
\bigl({{\overline{D_{23}}}^{y\eta}}\bigr)^2,
\notag
\end{equation}
%
\noindent
and $N$ is the Brunt-V\"ais\"al\"a frequency; the computation of $N$,
including moisture effects, is outlined in Section \ref{tke_section}.

Two options are available for calculating the mixing length $l_{h,v}$ in
%% \eqref{k_calc}.  An isotropic length scale can be chosen where $l_{h,v}
%% = (\Delta x \Delta y \Delta z)^{1/3}$ and thus $K_h = K_v = K$.  The
%% anisotropic option sets the horizontal mixing length $l_h = \sqrt{\Delta
\eqref{k_calc}.  An isotropic length scale (appropriate for $\Delta x, \Delta y
\simeq \Delta z$) can be chosen where $l_{h,v} = (\Delta x \Delta y \Delta z)^{1/3}$
and thus $K_h = K_v = K$.  The anisotropic option (appropriate for $\Delta x, 
\Delta y >> \Delta z$) sets the horizontal mixing length $l_h = \sqrt{\Delta
x \Delta y}$ in the calculation of the horizontal eddy viscosity $K_h$
using \eqref{k_calc}, and $l_v = \Delta z$ for the calculation of the
vertical eddy viscosity $K_v$ using \eqref{k_calc}.

Additionally, the eddy viscosities
for scalar mixing are divided by the turbulent Prandtl number $P_r = 1/3$.

\subsubsection{Prognostic TKE Closure}

For the predicted turbulent kinetic energy option
(TKE; see section \ref{tke_section}), the eddy viscosities are
computed using
%
\begin{equation}
K_{h,v} = C_k l_{h,v}\, \sqrt{e},
\notag
\end{equation}
%
\noindent
where $e$ is the turbulent kinetic energy (a prognostic variable 
in this scheme), $C_k$ is a constant (typically $0.15 < C_k < 0.25$),
and $l$ is a length scale.  

If the isotropic mixing option is chosen,
%
\begin{align}
l_{h,v} & = \hbox{min} 
\bigl[(\Delta x \Delta y \Delta z)^{1/3}, 0.76 \sqrt{e} / N\bigr]
\quad\quad\quad
\hbox{for } N^2 > 0, 
\notag \\
l_{h,v} & = (\Delta x \Delta y \Delta z)^{1/3} 
\,\,\,\,\,
\hphantom{hbox{min}\bigl[,0.76 \sqrt{e} / N\bigr]}
\hbox{for } N^2 \le 0
\notag 
\end{align} 
%
(see section \ref{tke_section} for the calculation of $N^2$).
Both the horizontal and vertical eddy viscosities are multiplied by an
inverse turbulent
Prandtl number $P_r^{-1} = 1 + 2l/(\Delta x \Delta y \Delta z)^{1/3}$ for
scalar mixing.  In this case $(l_{cr} > \Delta x)$ the 
horizontal and vertical eddy viscosities
are equivalent.

If the anisotropic mixing option is chosen, 
then $l_h=\sqrt{\Delta x \Delta y}$ for the calculation of $K_h$.
For calculating $K_v$, 
%
\begin{align} l_v &= \hbox{min} 
\bigl[\Delta z, 0.76 \sqrt{e} / N\bigr]
\quad\quad\quad \hbox{for } N^2 > 0,
\notag \\
l_v &= \Delta z \hphantom{\hbox{min} 
\bigl[, 0.76 \sqrt{e} / N\bigr]} \quad\quad\quad \hbox{for } N^2 \le 0.
\notag
\end{align} 
%
\noindent
The eddy viscosity used for mixing scalars is divided by a
turbulent Prandtl number $P_r$.  
The Prandtl number is 1/3 for the
horizontal eddy viscosity $K_h$,
and $P_r^{-1} = 1 + 2l/\Delta z$ for the
vertical eddy viscosity $K_v$.

\subsection{TKE equation for the 1.5 Order Turbulence Closure}
\label{tke_section}

The prognostic equation governing the evolution of the 
turbulent kinetic energy $e$ is 
%
\begin{equation}
 \partial_t (\mu_d e) + (\nabla \cdot {\bf V} e )_\eta 
= \mu_d (\hbox{ shear production } + \hbox{ buoyancy } + \hbox{ dissipation }).
\label{tke}
\end{equation}
%
\noindent
The time integration and the transport terms in \eqref{tke} are
integrated in the same manner as for other scalars (as described in
Chapter \ref{discretization_chap}). The right-hand side source and sink terms for $e$ are
given as follows.

\subsubsection{Shear Production}

The shear production term in \eqref{tke} can be written as
%
\begin{equation}
\hbox{shear production} =  K_h D_{11}^2
+ K_h D_{22}^2
+ K_v D_{33}^2
+ K_h {\overline{D_{12}^2}}^{xy}
+ K_v {\overline{D_{13}^2}}^{x\eta}
+ K_v {\overline{D_{23}^2}}^{y\eta}.
\notag
\end{equation}

\subsubsection{Buoyancy}

The buoyancy term in the TKE equation \eqref{tke} is written as
%
\begin{equation}
\hbox{ buoyancy} = - K_v N^2,
\notag
\end{equation}
%
\noindent
where the Brunt-V\"ais\"al\"a frequency $N$ is computed using either 
the formula for a moist saturated or unsaturated environment:
%
\begin{alignat}{2}
%% N^2 & = g \biggl[A {\partial \theta_e \over \partial z} 
N^2 & = g \bigl[A {\partial_z \theta_e } 
%% - {\partial q_w \over \partial z} \biggr] & \qquad
- {\partial_z q_w } \bigr] & \qquad
%% & \hbox{if $q_v \ge q_s$ or $q_c \ge 0.01$ g/Kg;} \notag \\
& \hbox{if $q_v \ge q_{vs}$ or $q_c \ge 0.01$ g/Kg;} \notag \\
%% N^2 & = g \biggl[{1 \over \theta}{\partial \theta \over \partial z}
N^2 & = g \biggl[{1 \over \theta}{\partial_z \theta}
%% + 1.61 {\partial q_v \over \partial z} - {\partial q_w \over \partial z}
+ 1.61 {\partial_z q_v } - {\partial_z q_w }
\biggr]
& \qquad 
%% & \hbox{if $q_v < q_s$ or $q_c < 0.01$ g/Kg}.
& \hbox{if $q_v < q_{vs}$ or $q_c < 0.01$ g/Kg}.
\notag
\end{alignat}
%
\noindent
The coefficient $A$ is defined as
%
\begin{equation}
A = \theta^{-1} 
{ 1 + {1.61\epsilon L q_v \over R_d T} \over
1 + {\epsilon L^2 q_v \over C_pR_v T^2} },
\notag
\end{equation}
%
\noindent
where $q_w$ represents the total water (vapor + all liquid species 
+ all ice species), $L$ is the latent heat of condensation and
%% $\epsilon$ is the molecular weight of water over the molecular weight of
$\epsilon$ is the ratio of the molecular weight of water vapor to the molecular weight of
dry air.  $\theta_e$ is the equivalent potential temperature and is
defined as 
%
\begin{equation}
%% \theta_e = \theta \biggl(1 + {\epsilon L q_{vs} \over C_p T} \biggr),
\theta_e = \theta \biggl(1 + { L q_{vs} \over C_p T} \biggr),
\notag
\end{equation}
%
\noindent
where $q_{vs}$ is the saturation vapor mixing ratio.

\subsubsection{Dissipation}

The dissipation term in \eqref{tke} is
%
\begin{equation}
\hbox{dissipation} = - {Ce^{3/2} \over l},
\notag
\end{equation}
%
\noindent
where 
%
\begin{equation}
C = 1.9 \, C_k  + { (0.93 - 1.9 \, C_k)\, l \over \Delta s},
\notag
\end{equation}
%
\noindent
$\Delta s =  (\Delta x \Delta y \Delta z)^{1/3}$,  
and
%
\begin{equation}
l = min\bigl[(\Delta x \Delta y \Delta z)^{1/3}, 0.76 \sqrt{e}/N\bigr].
\notag
\end{equation}

\subsection{Sixth-Order Spatial Filter on Coordinate Surfaces}

A sixth order spatial filter is available that is applied on horizontal
coordinate surfaces.  
The diffusion scheme is that proposed by \cite{xue.2000}.  Its application
in the ARW is described by \cite{knievel.et.al.2007}.  With the release of WRF Version 4, the 6th order filter was recast into conservative form and an optional limiter for diffusive fluxes based on the coordinate surface slope was introduced.  For the staggered velocities $u$ and $v$, and for the cell centered variables $a$, the filter can be expressed as
%
%\begin{equation}
%\partial_t (\mu_d a) = ... {\beta \, 2^{-6} \over 2 \Delta t} 
%\left[ \Delta x^6 \delta_x (\overline{\mu_d}^x F_x) + \Delta y^6 \delta_y (\overline{\mu_d}^y F_y)\right]
%\label{sixth_order_diffusion}
%\end{equation}
%
\begin{align}
\partial_t (\mu_d u) &= \cdots \, \beta {2^{-6} \over 2 \Delta t} 
\Big[ \Delta x^6 \, \overline{m_x}^x \delta_x \big({\mu_d} F_x(u)\big) + \Delta y^6  \,\overline{m_y}^x \delta_y \big(\overline{\mu_d}^{xy} F_y(u)\big)\Big],
\label{filter6_u} \\
%
\partial_t (\mu_d v) &= \cdots \, \beta {2^{-6} \over 2 \Delta t} 
\Big[ \Delta x^6 \, \overline{m_x}^y \delta_x \big(\overline{\mu_d}^{xy} F_x(v)\big) + \Delta y^6  \,\overline{m_y}^y \delta_y \big({\mu_d} F_y(v)\big)\Big],
\label{filter6_v} \\
%
\partial_t (\mu_d a) &= \cdots \, \beta {2^{-6} \over 2 \Delta t} 
\Big[ \Delta x^6 \, {m_x} \delta_x \big(\overline{\mu_d}^{x} F_x(a)\big) + \Delta y^6  \,{m_y} \delta_y \big(\overline{\mu_d}^y F_y(a)\big)\Big].
\label{filter6_a}
\end{align}
%
\noindent
$\beta$ is a user-specified filtering coefficient representing the
damping applied to $2\Delta$ waves for each filter application.  
The diffusive fluxes $\Delta x^5 F_x = \Delta x^5 \delta_x^5 (a)$, i.e.
%
\begin{equation} 
 \Delta x^5 F_x (a) {\Big|}_{i+1/2} = \gamma_s \Big\{ 10 \big[ a(i+1) - a(i) \big] - 5 \big[ a(i+2) - a(i-1) \big] + \big[ a(i+3) - a(i-2) \big] \Big\}.
\notag
\end{equation}
 %
\noindent
$\gamma_s$ is a flux limiting factor that reduces the diffusive flux where the terrain slope is steep.  $\gamma_s = 1$ if the limiter is inactive.  When activated by the user, the flux limiting factor
%
\begin{equation}
\gamma_s = \hbox{max} \,\left(1 - \left| \left({\partial z \over \partial x} \right) \left( {\partial z \over \partial x}\right)^{-1}_{cr} \right| , 0 \right).
\notag
\end{equation}
%
\noindent
$(dz/dx)$ is the slope of the coordinate surface, and it is approximated using the reference state geopotential in WRF.   $(dz/dx)_{cr}$ is a user specified critical value of the coordinate surface slope at which the diffusive flux is completely removed.  By design, $ 0 \le \gamma_s \le 1$.

The user can choose to enforce monotonicity in the filtering.  In the
monotonic option, the diffusive fluxes are set to zero if the diffusion is up-gradient, e.g. $F_x(a) = 0$ if $F_x (a) \, \delta_x(a) < 0$ and  $F_y(a) = 0$ if $F_y (a) \, \delta_y(a) < 0$.   

Note that the diffusive fluxes are computed in
computational space; map factors are not taken into account except in the flux divergence terms in (\ref{filter6_u}) - (\ref{filter6_a}) where they serve to guarantee scalar mass conservation.
The filter parameter $\beta$ is dimensionless, thus the filter
scales with the timestep and gridsize.  This explicit diffusion acts on all three components of wind,
potential temperature, all moisture variables and passive scalars, and
subgrid turbulence kinetic energy.  Further details can be found in
\cite{knievel.et.al.2007}.

%%%%%%%%%%%%%%%%%%%%%%%%%%
%
%  model filters that are not turbulence based
%
%%%%%%%%%%%%%%%%%%%%%%%%%%

\section{Filters for the Time-split RK3 scheme}

Three filters are used in the ARW time-split RK3
scheme apart from those in the model physics: three-dimensional
divergence damping (an acoustic model filter); an external-mode filter
that damps vertically-integrated horizontal divergence; and off-centering
of the vertically implicit integration of the vertical momentum
equation and geopotential equation.  Each of these is described in the
following sections.

\subsection{Three-Dimensional Divergence Damping}

The damping of the full mass divergence is a filter for acoustic
modes in the ARW solver.  This 3D mass divergence damping is described
in \citet{skamarock92}.  The filtering is accomplished by
forward weighting the pressure update in the acoustic step loop
described in Section \ref{full_time_split_integration}, step (6).  The linearized equation of
state \eqref{p-linear} is used to diagnose the pressure at the new time $\tau$
after the $U''$, $V''$, $\mu_d''$, and $\Theta_m''$ have
been advanced.  Divergence damping consists of using a modified pressure
in the computation of the pressure gradient terms in the horizontal
momentum equations in the acoustic steps (Equations \eqref{u-small-step}
and \eqref{v-small-step}).  Denoting the updated value as $p^\tau$, 
the modified pressure $p^{*\tau}$ used in \eqref{u-small-step}
and \eqref{v-small-step}
can be written as
%
\begin{equation}
p^{*\tau} = 
{p}^{\tau} + \gamma_d \bigl(
{p}^{\tau} -
{p}^{\tau - \Delta \tau}
\bigr),
\end{equation}
%
\noindent
where $\gamma_d$ is the damping coefficient.  
This modification is equivalent to inserting a horizontal 
diffusion term into the equation for the 3D mass divergence,
hence the name {\em divergence damping}.
A divergence damping coefficient $\gamma_d = 0.1$ is
typically used in the ARW
applications, independent of time step or grid size.

\subsection{External Mode Filter}

The external modes in the solution are damped by filtering
the vertically-integrated horizontal divergence.
This damping is accomplished by
adding a term to the horizontal momentum equations.  The additional
term added to \eqref{u-small-step} and \eqref{v-small-step} are
%
\begin{equation}
\delta_\tau U'' = ... \,\,
- \gamma_e {\Delta x^2 \over \Delta \tau} \delta_x (\delta_{\tau-\Delta \tau} \mu_d'')
\end{equation}
\noindent 
and
\begin{equation}
\delta_\tau V'' = ... \,\, 
- \gamma_e {\Delta y^2 \over \Delta \tau} \delta_y (\delta_{\tau-\Delta \tau} \mu_d'').
\end{equation}
%
\noindent
The quantity $\delta_{\tau - \Delta \tau} \mu $ is the vertically-integrated 
mass divergence (i.e., \eqref{omega}) from the previous
acoustic step (that is computed using
the time $\tau$ values of $U$ and $V$), and
$\gamma_e$ is the external mode damping coefficient.  An external mode
damping coefficient $\gamma_e = 0.01$  is typically used in the ARW
applications, independent of time step or grid size.

\subsection{Semi-Implicit Acoustic Step Off-centering}
\label{offcentering}

Forward-in-time weighting of the vertically-implicit
acoustic-time-step terms damps instabilities associated
vertically-propagating sound waves.  The forward weighting also damps
instabilities associated with sloping model levels and horizontally
propagating sound waves
\citep[see][]{durran_klemp83, dudhia95}.  The off-centering
is accomplished by using a positive (non-zero) coefficient $\beta$
\eqref{avg-operator} in the acoustic time-step vertical momentum
equation \eqref{w-small-step} and geopotential equation
\eqref{geo-small-step}.  An off-centering coefficient $\beta = 0.1$ is
typically used in the ARW applications, independent of time step or
grid size.

\section{Formulations for Gravity Wave Absorbing Layers}

There are three formulations for absorbing vertically-propagating
gravity waves so as to prevent unphysical wave reflection off the
domain upper boundary from contaminating ARW solutions.

\subsection{Absorbing Layer Using Spatial Filtering}
\label{absorbing_layer_spatial}

This formulation of the absorbing layer increases the second-order
horizontal and vertical eddy viscosities in the absorbing layer using
the following formulation:
%
\begin{equation}
K_{dh} = {\Delta x^2\over \Delta t} \gamma_g
\cos \biggl({\pi \over 2} \,{z_{top} - z \over z_{d}} \biggr),
\notag
\end{equation}
%
\noindent
and
%
\begin{equation}
K_{dv} = {\Delta z^2\over \Delta t} \gamma_g
\cos \biggl({\pi \over 2} \, {z_{top} - z \over z_{d}} \biggr).
\notag
\end{equation}
%
\noindent
Here $\gamma_g$ is a user-specified damping coefficient, $z_{top}$
is the height of the model top for a particular grid column, 
$z_d$ is the depth of the damping layer (from the model top), 
and $K_{dh}$ and $K_{dv}$ are the horizontal and vertical 
eddy viscosities for the wave absorbing layer.  If other 
spatial filters are being used, then the eddy viscosities
that are used for the second-order horizontal and vertical
eddy viscosities are the
maximum of ($K_h$, $K_{dh}$) and ($K_v$, $K_{dv}$).
The effect of this filter on gravity waves is discussed in
\citet{klemp_and_lilly78}, where guidance on the choice of 
the damping coefficient $\gamma_g$ can also be found.

\subsection{Implicit Rayleigh Damping for the Vertical Velocity}
\label{rayleigh-w}

Implicitly damping the vertical velocity within the implicit solution
algorithm for the vertically propagating acoustic modes has been found
to be a very effective and robust absorbing layer formulation by
\citet{klemp_et_al_2008}.  We recommend its use in both large-scale and
small-scale applications, and in idealized and real-data applications.
It has proven more effective than the spatial-filter-based absorbing
layer described in Section \ref{absorbing_layer_spatial}, and it is more
effective and more generally applicable than traditional Rayleigh
damping because of its implicit nature and the fact that it does not
need a reference state.  This formulation has been introduced in WRFV3.
It can only be used with the nonhydrostatic dynamics option.

In the vertically-implicit solution procedure for $W''^{\tau+\Delta
\tau}$ and $\phi''^{\tau+\Delta \tau}$ in the acoustic step (step 5 in
Figure \ref{time_integration_figure}), equation \eqref{geo-small-step}
is used to eliminate $\phi''^{\tau+\Delta \tau}$ from
\eqref{w-small-step} after which the tridiagonal equation in the
vertical direction for
$W''^{\tau+\Delta \tau}$ is solved.  Afterwards $\phi''^{\tau+\Delta \tau}$ is
recovered using \eqref{geo-small-step}.  In the solution procedure that
includes the implicit Rayleigh damping for $W$, after the tridiagonal
equation for $W''$ %$W''^{\tau+\Delta \tau}$ 
is solved and before the
geopotential $\phi$ is updated, the implicit Rayleigh damping is
included by calculating $W''^{\tau+\Delta \tau}$ using
%
\begin{equation}
W''^{\tau+\Delta \tau} = \tilde{W}''^{\tau+\Delta \tau} 
-\tau(z) \Delta \tau W''^{\tau+\Delta \tau}
\label{w-rayleigh}
\end{equation}
%
where $\tilde{W}''^{\tau+\Delta \tau}$ is the solution to
\eqref{w-small-step}.  The geopotential is then updated as usual using
\eqref{geo-small-step} with the updated damped vertical velocity from
\eqref{w-rayleigh}.
The perturbation pressure and specific volume
are computed as before using \eqref{p-linear} and \eqref{small-hydro}.

The variable $\tau(z)$ defines the vertical structure of the damping
layer, and has a form similar to the Rayleigh damper in
\citet{durran_klemp83} and also used in the traditional Rayleigh damping
formulation discussed in Section \ref{traditional-rayleigh}
%
\[ \tau (z) = \left\{ \begin{array}{cc}
         \gamma_r\sin^2 \left[ \frac{\pi}{2} 
\left( 1 - \frac{z_{top}-z}{z_d} \right) \right] & 
\mbox{for $z \geq (z_{top}-z_d)$};\\ 
        0 & \mbox{otherwise}, \end{array} \right. \] 
%
where $\gamma_r$ is a user specified damping coeficient, $z_{top}$ is
the height of the model top for a particular grid column, and $z_d$ is
the depth of the damping layer (from the model top).  A typical value
for the damping coefficient for this formulation is 
$\gamma_r = 0.2 \,s^{-1}$ ($\sim 10N$ for typical stratospheric values of
the Brunt-V\"ais\"all\"a frequency).
A complete
analysis of this filter can be found in \citet{klemp_et_al_2008}.

\subsection{Traditional Rayleigh Damping Layer}
\label{traditional-rayleigh}

A traditional Rayleigh damping layer is also available in the ARW solver.  This
scheme applies a tendency to $u$, $v$, $w$, and $\theta$ to gradually
relax the variable back to a predetermined reference state value,
%
\begin{eqnarray*}
%%   \frac{\partial u     }{\partial t} & = & -\tau (z) \left( u - \overline{u} \right) ,\\
%%   \frac{\partial v     }{\partial t} & = & -\tau (z) \left( v - \overline{v} \right) ,\\
%%    \frac{\partial w     }{\partial t} & = & -\tau (z)        w                        ,\\
%%   \frac{\partial \theta}{\partial t} & = & -\tau (z) \left( \theta - \overline{\theta} \right).
   {\partial_t u        } & = & -\tau (z) \left( u - \overline{u} \right) ,\\
   {\partial_t v        } & = & -\tau (z) \left( v - \overline{v} \right) ,\\
   {\partial_t w       } & = & -\tau (z)        w                        ,\\
   {\partial_t \theta} & = & -\tau (z) \left( \theta - \overline{\theta} \right).
\end{eqnarray*}
%
Overbars indicate the reference state fields, which are functions of $z$
only and are typically defined as the initial horizontally homogeneous
fields in idealized simulations.  The reference state vertical velocity
is assumed to be zero.  The vertical structure of the damping layer is
the same as that used for the implicit vertical velocity damping
described in Section \ref{rayleigh-w}.
Because the model surfaces change height with time in the ARW solver, the
reference state values at each grid point need to be recalculated at
every time step.  Thus, a linear interpolation scheme is used to
calculate updated reference state values based on the height of the
model levels at each time step.

The effect of this filter on gravity waves is discussed in
\citet{klemp_and_lilly78}, where guidance on the choice of 
the damping coefficeint $\gamma_r$ can also be found.

\section{Other Damping}

\subsection{Vertical-Velocity Damping}

This is also called $w$-damping.
In semi-operational or operational NWP applications,
the model robustness can be improved by detecting locations where the
vertical motion approaches the limiting Courant number for stability,
and applying a Rayleigh damping term in the vertical momentum equation to
stabilize the motion.  This term is non-physical and should
only be used in the situation where many, or long, model runs are being
done, and there is no option for a re-run with a shorter time-step if a 
failure occurs due to an excessively strong updraft. This might be the
case, for example, in an operational setting where real-time forecasts 
have to be produced on time to be useful. However, if this term activates
frequently, consideration should be given to reducing the model time-step.

The term is calculated from
%
\begin{equation}
Cr = \bigg | { \Omega dt \over \mu d\eta }\bigg |.
\notag
\end{equation}
%
\noindent
If $Cr > Cr_\beta$, then
\begin{equation}
%{\rm if} ~Cr > Cr_\beta ~ {\rm then:} ~~~~~
\partial_t W = ... - \mu_d ~{\rm sign} (W) \gamma_w ( Cr - Cr_\beta ),
\notag
\end{equation}
%
\noindent 
where $\gamma_w$ is the damping coefficient 
(typically 0.3 ms$^{-2}$), and $Cr_\beta$ 
is the activation Courant number (typically 1.0).  The ARW outputs
the location of this damping when it is active.
