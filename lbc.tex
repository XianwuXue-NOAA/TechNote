\chapter{Lateral Boundary Conditions}
\label{lbc_chap}

The ARW provides several lateral boundary conditions suitable 
for idealized flow, in addition to a specified lateral boundary
condition for real-data simulations.  These choices are
handled via a run-time option in the Fortran namelist file.
For nesting, all fine domains use the 
nest time-dependent lateral boundary condition where the outer row and column of the
fine grid is specified from the parent domain, described in 
Section \ref{nest-lbc}.
The remaining lateral boundary options are exclusively 
for use by the coarsest/parent domain.

\section{Periodic Lateral Boundary Conditions}

Periodic lateral boundary conditions in the ARW can be specified as
periodic in $x$ (west-east), $y$ (south-north), or doubly periodic in
$(x,y)$.  The periodic boundary conditions constrain the solutions to be
periodic; that is, a generic model state variable $\psi$ will follow
the relation 
% 
\begin{equation} \psi(x+nL_x,y+mL_y) = \psi(x,y)
%\label{periodic} 
\notag
\end{equation} 
% 
\noindent for each integer $(n,m)$.
The periodicity lengths $(L_x,L_y)$ are \hfil\break 
[(dimension of the domain in $x$) - 1]$\Delta x$ and
[(dimension of the domain in $y$) - 1]$\Delta y$.

\section{Open Lateral Boundary Conditions}

Open lateral boundary conditions, also referred to as gravity-wave
radiating boundary conditions, can be specified for the west, east, 
north, or south boundary, or any combination thereof.  
The gravity wave radiation conditions follow the approach of 
\citet{klemp_and_lilly78} and \citet{klemp78}.

There are a 
number of changes in the base numerical algorithm 
described in Chapter \ref{discretization_chap} 
that accompany the imposition of these
conditions.  First, for the normal horizontal velocities along
a boundary on which the condition is specified, the momentum
equation for the horizontal velocity, \eqref{u-small-step} or 
\eqref{v-small-step}, is replaced by
%
\begin{equation}
\delta_\tau U'' = - U^* \delta_x u,
\notag
\end{equation}
%
\noindent
where $U^* = \hbox{min}(U - c_b,0)$ at the $x = 0$ (western) boundary,
$U^* = \hbox{max}(U + c_b,0)$ at the $x = L$ (eastern) boundary, and
likewise for the south and north boundaries for the $V$ momentum.  The
horizontal difference operator $\delta_x$ is evaluated in a one-sided
manner using the difference between the boundary value and the
value one grid-point into the grid, from the boundary.  $c_b$ is the phase
speed of the gravity waves that are to be radiated; it is specified
as a model constant 
\citep[for more details see][]{klemp_and_lilly78, klemp78}.
%\citep{klemp78}.

For scalars and non-normal momentum variables, the
boundary-perpendicular flux divergence term is replaced with
%
\begin{equation}
\delta_x (U\psi) = U^* \delta_x \psi + \psi \ \delta_x U,
\notag
\end{equation}
%
\noindent
where $U^* = \hbox{min}(U,0)$ at the $x = 0 + \Delta x/2$ (western)
scalar boundary,
$U^* = \hbox{max}(U,0)$ at the $x = L - \Delta x/2$ (eastern) boundary, and
likewise for the south and north boundaries, using $V$.  As was the case
for the momentum equations, the
horizontal difference operator $\delta_x$ is evaluated in a one-sided
manner using the difference between the boundary value and the
value one grid-point into the grid from the boundary.

\section{Symmetric Lateral Boundary Conditions}

Symmetry lateral boundary conditions can be specified for the west,
east, north, or south boundary, or any combination thereof.  The
symmetry boundaries are located on the normal-velocity planes at the
lateral edges of the grids.  The normal velocities are zero at
these boundaries, and on either side of the boundary the normal 
velocity satisfies the relation
%
\begin{equation}
U_{\perp}(x_b - x) = - U_{\perp}(x_b + x),
%\label{symmetry_u}
\notag
\end{equation}
%
\noindent
where $x_b$ is the location of the symmetry boundary.
All other variables satisfy the relation
\begin{equation}
\psi(x_b - x) = \psi(x_b + x).
%\label{symmetry}
\notag
\end{equation}


\section{Specified Lateral Boundary Conditions}
\label{lbc_spec}


%
% spec and relax zones for  real-data lbc
%
\begin{figure}
 \centering
  \includegraphics *[width=4.5in]{figures/lbc_zones.pdf}
  \caption{\label{figure:spec} Specified and relaxation zones for a grid
with a single-specified row and column, plus four rows and columns for the
relaxation zone.  These are typical values used for a specified lateral boundary
condition for a real-data case.}
\end{figure}


Primarily for real-data cases, the specified boundary condition is also referred to as a
relaxation, or nudging, boundary condition.  There are two uses of
specified boundaries in the ARW: for the outer-most coarse grid or
for the time-dependent boundaries supplied to a nested grid.
The specified lateral boundary conditions for the nest are automatically selected for all of the
fine grids, even if the coarse grid is using combinations
of the symmetry, periodic, or open options.  
If the specified lateral boundary condition is selected for the coarse
grid, then all four grid
sides (west, east, north, and south) use specified lateral conditions.
However, in tropical channel mode, where the domain wraps completely around
the equator, it is possible to combine specified boundary conditions with
periodic conditions in the $x$-direction. Note that care is needed in setting
up the domain so that the points exactly match the longitude at the east and west
boundaries when periodic conditions are used in real-data cases. Also, note that
a Mercator projection is needed to make this possible.

The coarse grid specified lateral boundary is comprised of both a specified and
a relaxation zone (as shown in Fig. \ref{figure:spec}).  
For the coarse grid, the specified zone is determined entirely by 
temporal interpolation from an external
forecast or analysis (supplied by WPS).  The width of the specified zone is run-time
configurable, but is typically set to 1 (i.e., the last
row and column along the outer edge of the coarsest grid
is entirely specified by temporal interpolation, using data
from an external model).
The second region of the lateral boundary for the coarse grid is the relaxation zone,
which is where the model is nudged or relaxed
towards the large-scale forecast (e.g., rows and columns 2 through 5 in 
Fig. \ref{figure:spec}).
The size of the relaxation zone is a run-time option.

The specified lateral boundary condition for the coarse grid 
requires an external file, generated
during the same pre-processing as the initial condition file.
Let $\psi$ be any prognostic value having a lateral boundary entry, after \cite{daviesturner77},

\begin{equation}
\partial_t \psi \big|_n = F_1(\psi_{LS} - \psi) - F_2 \Delta^2(\psi_{LS} - \psi),
\label{lbc_relax}
\end{equation}
\noindent where $n$ is the number of grid points in from the outer row or column along the boundary
($SpecZone + 1 \leq n \leq SpecZone + RelaxZone - 1$; see Fig. \ref{figure:spec}) 
and $\psi_{LS}$ is the large-scale value obtained by spatial and temporal interpolation from the external analysis or model forecast by WPS .  $\Delta^2$
is a 5-point horizontal smoother applied along $\eta$-surfaces.
The weighting function
coefficients $F_1$ and $F_2$ in \eqref{lbc_relax} are given by
\begin{equation}
F_1 = {1 \over {10 \Delta t }} {{SpecZone + RelaxZone - n} \over {RelaxZone - 1}},
\notag
\end{equation}
\begin{equation}
F_2 = {1 \over {50 \Delta t }} {{SpecZone + RelaxZone - n} \over {RelaxZone - 1}},
\notag
\end{equation}
\noindent where $n$ extends only through the relaxation zone 
($SpecZone + 1 \leq n \leq SpecZone + RelaxZone - 1$).
$F_1$ and $F_2$ are
linear ramping functions with a maximum at the first relaxation row or column 
nearest to the coarse grid boundary (just inside the specified zone).  
In Version 3.0 these linear functions can optionally be multiplied by an
exponential function for smoother behavior when broad boundary zones are
needed. This is achieved through an additional multiplier.
\begin{equation}
F_1 = {1 \over {10 \Delta t }} {{SpecZone + RelaxZone - n} \over {RelaxZone - 1}}
\exp [ - SpecExp ( n - SpecZone - 1)]
\notag
\end{equation}
\begin{equation}
F_2 = {1 \over {50 \Delta t }} {{SpecZone + RelaxZone - n} \over {RelaxZone - 1}}
\exp [ - SpecExp ( n - SpecZone - 1)]
\notag
\end{equation}
where $SpecExp$ is an inverse length scale in grid lengths.

On the coarse grid, the specified boundary condition applies to 
the horizontal wind components, potential temperature, $\phi'$, $\mu'_d$, and water vapor. 
The lateral boundary file contains enough information to update
boundary zone values through the entire simulation period.  The momentum fields are 
coupled with $\mu_d$ and the inverse map factor (both at the specific variable's
horizontal staggering location), and the other 3-dimensional fields are coupled with
$\mu_d$.  The $\mu'_d$ field is not coupled in the lateral boundary file.

Vertical velocity has a zero gradient boundary condition applied in the
specified zone (usually the outer-most row and column).
Microphysical variables (with the exception of vapor) and all other scalars have flow-dependent boundary
conditions applied in the specified zone. This boundary condition is specified as zero on inflow
and zero-gradient on outflow. Since these boundary conditions require only information from
the interior of the grid, these variables are not in the specified boundary condition file.
However, for nested boundaries, microphysical variables and scalars are also available
from the parent domain, and use the same relaxation towards interpolated variables as the
dynamic variables.

\section{Polar Conditions}
See subsection \ref{pole_condition} for details on how the polar boundary
condition is applied.
