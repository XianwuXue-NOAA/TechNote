\chapter{Model Discretization}
\label{discretization_chap}

\section{Temporal Discretization}

The ARW solver uses a time-split integration scheme.  Generally
speaking, slow or low-frequency (meteorologically significant) modes are
integrated using a third-order Runge-Kutta (RK3) time integration
scheme, while the high-frequency acoustic modes are integrated over
smaller time steps to maintain numerical stability.  The horizontally
propagating acoustic modes (including the external mode present in the
mass-coordinate equations using a constant-pressure upper boundary
condition) and gravity waves are integrated using a forward-backward
time integration scheme, and vertically propagating acoustic modes and
buoyancy oscillations are integrated using a vertically implicit scheme
(using the acoustic time step).  The time-split integration for the
flux-form equations is described and analyzed in
\citet{klemp_et_al_2007}.  The time-splitting is similar to that first
developed by \citet{klemp78} for leapfrog time integration and analyzed
by \citet{skamarock92}.  This time-split approach was extended to the
RK3 scheme as described in \citet{wicker02}.  The primary differences
between the earlier implementations described in the references and the
ARW implementation are associated with our use of the mass vertical
coordinate and a flux-form set of equations, as described in
\citet{klemp_et_al_2007}, along with our use of
perturbation variables for the acoustic component of the time-split
integration.  The acoustic-mode integration is cast in the form of a
correction to the RK3 integration.

\subsection{Runge-Kutta Time Integration Scheme}
\label{rk3_scheme}

The RK3 scheme, described in \citet{wicker02},
integrates a set of ordinary differential equations using a
predictor-corrector formulation.  Defining the prognostic variables in
the ARW solver as $\Phi = (U,V,W,\Theta_m,\phi', \mu_d', Q_m)$ and the 
model equations as $\Phi_t = R(\Phi)$, the RK3
integration takes the form of 3 steps to advance a solution 
$\Phi(t)$ to $\Phi(t+\Delta t)$:
%
\begin{align}
\Phi^* & = \Phi^t + {\Delta t \over 3} R(\Phi^t) 
\label{rk3a}
\\
\Phi^{**} & = \Phi^t + {\Delta t \over 2} R(\Phi^*) 
\label{rk3b}
\\
\Phi^{t+\Delta t} & = \Phi^t + {\Delta t} R(\Phi^{**})
\label{rk3c}
\end{align}
%
\noindent
where $\Delta t$ is the time step for the low-frequency modes
(the model time step).
In \eqref{rk3a} -- \eqref{rk3c}, superscripts denote time levels.
This scheme is not a true Runge-Kutta scheme {\it per se} because, while it is 
third-order accurate for linear equations, it is only second-order accurate
for nonlinear equations.  With respect to the ARW equations, the 
time derivatives $\Phi_t$ are the partial time derivatives (the leftmost
terms) in equations \eqref{tm_full}, \eqref{qm_full}, and \eqref{u-mom-pert} -- \eqref{phi_pert}, and
$R(\Phi)$ are the remaining terms in those equations.

\subsection{Acoustic Integration}
\label{acoustic_equations}

The high-frequency but meteorologically insignificant acoustic modes
would severely limit the RK3 time step $\Delta t$ in \eqref{rk3a} --
\eqref{rk3c}.  To circumvent this time step limitation we use the 
time-split approach described in \citet{wicker02}.  Additionally, to increase the
accuracy of the splitting, we integrate 
a perturbation form of the governing equations using smaller acoustic
time steps within the RK3 large-time-step sequence.
To form the perturbation
equations for the RK3 time-split acoustic integration, we define small
time step variables that are deviations from the most recent RK3
predictor (denoted by the superscript $t^*$ and representing either
$\Phi^t$, $\Phi^*$, or $\Phi^{**}$ in \eqref{rk3a} -- \eqref{rk3c}): 
%
\begin{align} 
\hbox{\bf V}'' &= \hbox{\bf V} -\hbox{\bf V}^{t^*}, ~~~~~~
\Omega'' = \Omega - \Omega^{t^*}, ~~~~~~ \Theta_m'' = \Theta_m -
\Theta_m^{t^*}, \notag \\ \phi'' &= \phi' - {\phi'}^{t^*}, 
~~~~~~ \alpha_d''= \alpha_d' - {\alpha_d'}^{t^*}, ~~~~~~
\mu_d''= \mu_d' - {\mu'}_d^{t^*}.  \notag 
\end{align} 
% 
\noindent 
The hydrostatic relation (i.e., the vertical coordinate definition) becomes 
% 
\begin{equation}
\alpha_d'' = -{1\over\mu_d^{t^*}}\Bigl(\partial_\eta \phi'' + \alpha_d^{t^*}\mu_d''
\Bigr).  
\label{small-hydro} 
\end{equation} 
% 
\noindent
Additionally, we also
introduce a version of the equation of state that is linearized about $t^*$, 
%
\begin{equation} p'' = {c_s^2\over\alpha_d^{t^*}} \biggl({\Theta_m''\over\Theta_m^{t^*}}
-{\alpha_d''\over\alpha_d^{t^*}} - {\mu_d''\over\mu_d^{t^*}}\biggr), 
\label{p-linear}
\end{equation} 
% 
\noindent 
where $c_s^2 = \gamma p^{t^*} \alpha_d^{t^*}$ is the
square of the sound speed.
The linearized
state equation \eqref{p-linear} and the vertical coordinate definition
\eqref{small-hydro} are used to cast the vertical pressure gradient in
\eqref{w-mom-pert} in terms of the model's prognostic variables.
By combining \eqref{p-linear} and \eqref{small-hydro}, the vertical
pressure gradient can be expressed as
%
\begin{equation}
  \partial_\eta p'' = 
     \partial_\eta (C \partial_\eta \phi'') 
   +
\partial_\eta\biggl({c_s^2\over\alpha_d^{t^*}}
{\Theta_m''\over\Theta_m^{t^*}}\biggr),
\label{pz-linear}
\end{equation}
%
\noindent
where $C=c_s^2/\mu_d^{t^*}{\alpha^{t^*}_d}^2$. This linearization about the most
recent large time step should be highly accurate over the time interval of
the several small time steps.

These variables along with \eqref{pz-linear} 
are substituted into equations
\eqref{tm_full} and \eqref{u-mom-pert} -- \eqref{hydro-pert} and lead to the 
acoustic time-step equations:
%
\begin{align}
  \null\hskip-.5in
\partial_t U'' + (m_x/m_y)(\alpha^{t^*}/\alpha^{t^*}_d) \left[\mu_d^{t^*} \left(
\alpha_d^{t^*} \partial_x {p''}^\tau +  {\alpha_d''}^\tau \partial_x {\overline p}  + \partial_x {\phi''}^\tau \right)
+ \partial_x \phi^{t^*} \left(\partial_\eta {p''} - {\mu''_d} \right)^\tau \right] & = R_U^{t^*}
%% \delta_\tau U'' + \mu^{t^*}\alpha^{t^*} \partial_x {p''}^\tau 
%%                 + (\mu^{t^*} \partial_x \bar p){\alpha''}^\tau  
%%                 + (\alpha/\alpha_d) [\mu^{t^*} \partial_x {\phi''}^\tau 
%%                 + (\partial_x \phi^{t^*})(\partial_\eta p''-\mu'')^\tau]
%%                                                   &= {R_U}^{t^*}  
\label{u-small-step} 
\\
  \null\hskip-.5in
\partial_t V'' + (m_y/m_x)(\alpha^{t^*}/\alpha^{t^*}_d) \left[\mu_d^{t^*} \left(
\alpha_d^{t^*} \partial_y {p''}^\tau +  {\alpha_d''}^\tau \partial_y {\overline p}  + \partial_y {\phi''}^\tau \right)
+ \partial_y \phi^{t^*} \left(\partial_\eta {p''} - {\mu''_d} \right)^\tau \right] & = R_V^{t^*}
%% \delta_\tau V'' + \mu^{t^*}\alpha^{t^*} \partial_y {p''}^\tau 
%%                 + (\mu^{t^*} \partial_y \bar p){\alpha''}^\tau  
%%                 + (\alpha/\alpha_d) [\mu^{t^*} \partial_y {\phi''}^\tau 
%%                 + (\partial_y \phi^{t^*})(\partial_\eta p''-\mu'')^\tau]
%%                                                   &= {R_V}^{t^*}   
\label{v-small-step}
\\
 \delta_\tau \mu_d'' 
 + m_x m_y[\partial_x U'' + \partial_y V'']^{\tau+\Delta \tau} 
+ m_y \partial_\eta {\Omega''^{\tau+\Delta \tau}} &= {R_\mu}^{t^*}  
\label{mu-small-step}
\\
 \delta_\tau \Theta_m'' 
+ m_x m_y[  \partial_x (U''\theta_m^{t^*}) 
      + \partial_y (V''\theta_m^{t^*})]^{\tau+\Delta \tau} 
+ m_y \partial_\eta (\Omega''^{\tau+\Delta \tau} \theta_m^{t^*})
                                                &= {R_{\Theta_m}}^{t^*}  
\label{theta-small-step}
\\
 \delta_\tau W'' 
- m_y^{-1} g\overline{\left\{(\alpha/\alpha_d)^{t^*} \left[
     \partial_\eta (C \partial_\eta \phi'') 
   +
\partial_\eta\biggl({c_s^2\over\alpha^{t^*}}{\Theta_m''\over\Theta_m^{t^*}}\biggr)
\right]
- \mu_d''\right\}}^\tau
                                                   &= {R_W}^{t^*}  
\label{w-small-step}
\\
 \delta_\tau \phi'' + {1\over\mu_d^{t^*}} 
%% [ m_y \Omega''^{\tau+\Delta \tau}\phi_\eta^{t^*}  - m_y \overline{g W''}^\tau ] 
[ m_y \Omega''^{\tau+\Delta \tau}\delta_\eta \phi^{t^*}  - m_y \overline{g W''}^\tau ] 
&= {R_\phi}^{t^*}.
\label{geo-small-step}
\end{align}
%
\noindent
The RHS terms in \eqref{u-small-step} -- \eqref{geo-small-step} are
fixed for the acoustic steps that comprise the time integration of each
RK3 sub-step (i.e., \eqref{rk3a} -- \eqref{rk3c}),
and are given by
%
\begin{align}
R_U^{t^*} = & - m_x\left[\partial_x (Uu) + \partial_y (Vu)\right] - \partial_\eta (\Omega u)  & \cr
& -(m_x/m_y) (\alpha/\alpha_d) \left[ \mu_d (\partial_x \phi' + \alpha_d \partial_x p' + \alpha'_d \partial_x \overline{p}) +
\partial_x \phi (\partial_\eta p' - \mu'_d)\right]
%%R_U^{t^*} = &
%%- m_x[\partial_x(Uu) + \partial_y(Vu)] - \hphantom{(m_y/m_x)} \partial_\eta (\Omega u)
%%- ({\mu}_d \alpha \partial_x p' 
%%- {\mu}_d \alpha' \partial_x \bar{p}) ~~~~~~~~ \notag
%%\\
%%& - (\alpha/\alpha_d) ( {\mu}_d \partial_x \phi' 
%%  -  \partial_\eta p' \partial_x \phi 
%%  + {\mu}_d' \partial_x \phi ) + F_U
\label{u-rhs}
\\
%
R_V^{t^*} = & - m_y\left[\partial_x (Uv) + \partial_y (Vv)\right] - (m_y/m_x) \partial_\eta (\Omega v)  & \cr
& -(m_y/m_x) (\alpha/\alpha_d) \left[ \mu_d (\partial_y \phi' + \alpha_d \partial_y p' + \alpha'_d \partial_y \overline{p}) +
\partial_y \phi (\partial_\eta p' - \mu'_d)\right]
%%R_V^{t^*} = &
%%- m_y[\partial_x (Uv) + \partial_y (Vv)] - (m_y/m_x) \partial_\eta (\Omega v)
%%- ({\mu}_d \alpha \partial_y p' 
%%- {\mu}_d \alpha' \partial_y \bar{p}) ~~~~~~~~ \notag
%%\\
%%& - (\alpha/\alpha_d) ( {\mu}_d \partial_y \phi' 
%%  - \partial_\eta p' \partial_y \phi 
%%  + {\mu}_d' \partial_y \phi ) + F_V
\label{v-rhs}
\\ 
%
R_{\mu_d}^{t^*} = &
- m_x m_y[\partial_x U + \partial_y V] - m_y \partial_\eta \Omega
\label{mu-rhs}
\\
R_{\Theta_m}^{t^*} = &
- m_x m_y [\partial_x (U\theta_m) + \partial_y (V\theta_m)] - m_y \partial_\eta
(\Omega \theta_m) + F_{\Theta_m}
\label{theta-rhs}
\\
%
R_W^{t^*} = &
- m_x [\partial_x (Uw) + \partial_y (Vw)] - \partial_\eta
(\Omega w)  ~~~~~~~~ ~~~~~~~~ ~~~~~~~~ ~~~~~~~~ ~~~ \notag 
\\
& + m_y^{-1} g (\alpha/\alpha_d) [\partial_\eta p' 
  - {\bar{\mu}}_d (q_v + q_c +q_r)]
  - m_y^{-1} {\mu}_d'g + F_W
\label{w-rhs} 
\\
%
R_\phi^{t^*} = &
- \mu_d^{-1}
%% [m_x m_y (U\phi_x + V\phi_y) + m_y
%% \Omega\phi_\eta - m_y gW ],
[m_x m_y (U\partial_x\phi + V\partial_y\phi) + m_y
\Omega\partial_\eta\phi - m_y gW ],
\label{geo-rhs}
%
\end{align}
%
\noindent
where all variables in \eqref{u-rhs} -- \eqref{geo-rhs} are evaluated at
time $t^*$ (i.e., using $\Phi^t$, $\Phi^*$, or $\Phi^{**}$ for the
appropriate RK3 sub-step in \eqref{rk3a} -- \eqref{rk3c}).  
Equations \eqref{u-small-step} -- \eqref{geo-small-step} utilize
the discrete
acoustic time-step operator
%
\begin{equation}
\delta_\tau a = { a^{\tau + \Delta \tau} - a^\tau \over \Delta \tau },
\notag
\end{equation}
%
\noindent
where $\Delta \tau$ is the acoustic time step, 
and terms averaged in time over an acoustic time step are slightly forward centered
using an averaging operator 
%
\begin{equation}
\overline{a}^\tau = {1 + \beta \over 2}a^{\tau+\Delta \tau} +
{1 - \beta \over 2}a^{\tau},
\label{avg-operator}
\end{equation}
%
\noindent
where $\beta$ is a user-specified parameter (see Section \ref{offcentering}).

The integration over the acoustic time steps proceeds as follows.
Beginning with the small time-step variables at time $\tau$, 
\eqref{u-small-step}
and \eqref{v-small-step}
are stepped forward to obtain ${U''}^{\tau+\Delta\tau}$ and
${V''}^{\tau+\Delta\tau}$. Both ${\mu''}^{\tau+\Delta\tau}$ and
${\Omega''}^{\tau+\Delta\tau}$ are then calculated from
\eqref{mu-small-step}. 
This is
accomplished by first integrating \eqref{mu-small-step} 
vertically from the surface to the
material surface at the top of the domain, which removes the $\partial_\eta
\Omega''$ term. Recalling that $\mu_d=\partial p_d/\partial\eta$ and that $p_d$
for the hybrid vertical coordinate is defined by \eqref{hyb_def}, the vertical 
integral of \eqref{mu-small-step} becomes,
%
\begin{equation}
  \delta_\tau{p_c}'' = m_x m_y \int_1^0 
 [\partial_x U'' + \partial_y V'']^{\tau+\Delta \tau} 
d\eta,
\label{omega}
\end{equation}
%
where $p_c(x,y)=p_s-p_t$ is the dry hydrostatic pressure difference (mass) in the
vertical column at $(x,y)$.
After computing $\delta_\tau{p_c''}$ from \eqref{omega},
${\Omega''}^{\tau+\Delta\tau}$ is obtained by vertically integrating the
$\partial_\eta \Omega''$ term \eqref{mu-small-step} (with $\delta_\tau\mu=B_\eta\delta _\tau p_c$) using the lower boundary 
condition $\Omega''=0$ at the
surface, which yields
 %
\begin{equation}
{\Omega''}^{\tau+\Delta\tau} = m_y^{-1}\bigl[1-B(\eta)\bigr]\delta_\tau p_c- m_x\int_1^\eta 
 [\partial_x U'' + \partial_y V'']^{\tau+\Delta \tau} 
d\eta,
\label{omega_int}
\end{equation}
%
and $\mu_d''^{\tau+\Delta\tau}$ is recovered using \eqref{hyb_def}: 
 %
\begin{equation}
{\mu_d''}^{\tau+\Delta\tau}(x,y,\eta) = B_\eta(\eta){p_c''}^{\tau+\Delta\tau}(x,y)+\bigl[1-B_\eta(\eta)\bigr](p_0-p_t).
\label{mu_from_pc}
\end{equation}
%
From \eqref{mu_from_pc}, it is evident that $\mu$ need not be stored as a three-dimensional array, but can readily constructed when needed from the two-dimensional $p_c(x,y)$ array together with the one-dimensional $B_\eta(\eta)$ profile.

Knowing ${\Omega''}^{\tau+\Delta\tau}$, \eqref{theta-small-step} can be stepped forward to calculate
${\Theta_m''}^{\tau+\Delta\tau}$. Equations \eqref{w-small-step} and \eqref{geo-small-step}
are combined to form a vertically implicit equation that is solved for
${W''}^{\tau+\Delta\tau}$ subject to the boundary condition
%$W''={\hbox{\bf V}''}\cdot{\nabla}h$ 
$\Omega = \Omega'' = 0$
at the surface ($z=h(x,y)$) and $p'=0$ 
along the model top. ${\phi''}^{\tau+\Delta\tau}$ is  then
obtained from \eqref{geo-small-step}, and ${p''}^{\tau+\Delta\tau}$ and 
${\alpha_d''}^{\tau+\Delta\tau}$ are recovered from \eqref{p-linear}
and \eqref{small-hydro}.
 
\noindent
\begin{figure}  % [h]
\setlength{\fboxrule}{.75pt}
\framebox[\columnwidth]{
\parbox{6.5truein}{
\vskip 5truept
\noindent
{\bf Begin Time Step} \medskip \hfill \break
%
\hphantom{Begin} {\bf Begin RK3 Loop: Steps 1, 2, and 3} \smallskip\hfill \break
%
\hphantom{BeginBegin} 
(1) If RK3 step 1, compute and store $F_\Phi$ \hfill
\break
%
\hphantom{BeginBegin If RK3 step 1,}
(i.e., physics tendencies for RK3 step, including
mixing). \smallskip \hfill \break
%
\hphantom{BeginBegin} 
%% (2) Compute $R_\Phi^{t^*}$,
(2) Compute $R^{t^*}$,
\eqref{u-rhs}--\eqref{geo-rhs}
\medskip \hfill \break
%
\hphantom{BeginBegini}{\bf Begin Acoustic Step Loop: Steps $1 \to n$},
\hfill \break
%
\hphantom{BeginBeginBegini} 
{\it RK3 step 1, $n = 1$, ~~~~$\Delta \tau = \Delta t/3$; \hfill \break
\hphantom{BeginBeginBegini} 
RK3 step 2, $n = n_s/2$, $\Delta \tau = \Delta t/n_s$; \hfill \break
\hphantom{BeginBeginBegini} 
RK3 step 3, $n = n_s$, ~~\,\,$\Delta \tau = \Delta t/n_s$.}
\medskip \hfill \break
%
\hphantom{BeginBeginBegini}
(3) Advance horizontal momentum, \eqref{u-small-step} 
and \eqref{v-small-step} \hfill\break
%
\hphantom{BeginBeginBegini(3) }
Global: Apply polar filter to $U''^{\tau+\Delta \tau}$ 
and $V''^{\tau+\Delta \tau}$.
\hfill\break
%
\hphantom{BeginBeginBegini}
(4) Advance $\mu_d$ \eqref{mu-small-step} and compute 
${\Omega''}^{\tau+\Delta \tau}$ then
advance $\Theta_m$ 
\eqref{theta-small-step} \hfill\break
\hphantom{BeginBeginBegini(4) }
Global: Apply polar filter to $\mu_d^{\tau+\Delta \tau}$ 
and $\Theta_m''^{\tau+\Delta \tau}$. \hfill\break
%
\hphantom{BeginBeginBegini}
(5) Advance $W$ and $\phi$ \eqref{w-small-step} 
and \eqref{geo-small-step} \hfill\break
\hphantom{BeginBeginBegini(5) }
Global: Apply polar filter to $W''^{\tau+\Delta \tau}$
and $\phi''^{\tau+\Delta \tau}$. \hfill\break
%
\hphantom{BeginBeginBegini}
(6) Diagnose $p''$ and $\alpha''$ using
\eqref{p-linear} and  \eqref{small-hydro}
\medskip \hfill \break
%
\hphantom{BeginBegini}{\bf End Acoustic Step Loop}
\medskip
\hfill \break
%
\hphantom{BeginBegin} 
(7) Scalar transport: Advance scalars \eqref{qm_full} 
\hfill \break \hphantom{BeginBeginBegin} over RK3
substep \eqref{rk3a}, \eqref{rk3b} or \eqref{rk3c} \hfill \break
\hphantom{BeginBeginBegin} 
(using mass fluxes $U$, $V$ and $\Omega$ time-averaged over the acoustic steps).
\hfill \break
\hphantom{BeginBeginBegin}
Global: Apply polar filter to scalars. \smallskip\hfill\break
%
\hphantom{BeginBegin} 
(8) Using updated prognostic variables, compute $p'$ with \eqref{ideal_gas_law} and $\alpha'$ with
\hfill \break \hphantom{BeginBeginBegin} \eqref{hydro-pert} or \eqref{hypsometric_eqn}
\medskip \hfill \break
%
\hphantom{Begin} {\bf End RK3 Loop} 
\medskip \hfill \break
%
\smallskip
\hphantom{Begin} 
(9) Compute non-RK3 physics (currently microphysics), advance variables.
\hfill \break
\hphantom{Begin(4) }
Global: Apply polar filter to updated variables. \medskip\hfill\break
%
{\bf End Time Step}
\vskip 5truept
}
}
\caption{Time step integration sequence. Here $n$ represents the number
of acoustic time steps for a given substep of the RK3 integration, and 
$n_s$ is the ratio of the RK3 time step to the acoustic time step for the
second and third RK3 substeps.}
\label{time_integration_figure}
\end{figure}

\subsection{Full Time-Split Integration Sequence}
\label{full_time_split_integration}

The time-split RK3 integration technique is summarized in Figure \ref{time_integration_figure}.
It consists of two primary loops--- an outer loop for the
large-time-step Runge-Kutta integration, and an inner loop for the
acoustic mode integration.  

In the RK3 scheme, physics can be integrated within the RK3 time
integration (using a time forward step, i.e., step (1) in Fig.
\ref{time_integration_figure}, or the RK3 time integration if higher
temporal accuracy is desired, i.e., in step (2)--- implying a physics
evaluation every RK3 substep) or external to it using additive
timesplitting, i.e., step (9).

Within the acoustic integration, the acoustic time step $\Delta \tau$ is
specified by the user through the choice of $n_s$ (see Section
\ref{acoustic_step_constraint}).  Within the first RK3 substep, however,
a single acoustic time step is used to advance the solution regardless of
$n_s$.  Within the full RK3-acoustic timesplit integration, this
modified acoustic time step does not impose any additional stability
constraints \citep[see][]{wicker02}.

The major costs in the model arise from the evaluation of the 
right hand side terms $R^{t^*}$ in \eqref{u-small-step}
-- \eqref{geo-small-step}.
The efficiency of the RK3 timesplit
scheme arises from the fact that the RK3 time step $\Delta t$ is much
larger than the acoustic time step $\Delta \tau$, hence the most costly
evaluations are only performed in the less-frequent RK3 steps.

\subsection{Diabatic Forcing}
\label{diabatic forcing subsection}

Within the RK3 integration sequence outlined in Fig.
\ref{time_integration_figure}, the RHS term $R_{\Theta_m}^{t^*}$ in the
thermodynamic equation \eqref{theta-small-step} contains contributions
from the diabatic physics tendencies that are computed in step (1) at
the beginning of the first RK3 step.  This diabatic forcing is
integrated within the acoustic steps (specifically, in step 4 in the
time integration sequence shown in Fig. \ref{time_integration_figure}).
Additional diabatic contributions
are integrated in an additive-time-split manner in step (9) after the RK3
update is complete.  Thus, the diabatic forcing computed in step (9) (the
microphysics in the current release of the ARW) does not appear in
$R_{\Theta_m}^{t^*}$ from \eqref{theta-small-step} in the acoustic
integration.  We have discovered that this time splitting can excite
acoustic waves and can give rise to noise in the solutions for some
applications.  Note that the non-RK3 physics are integrated in step (9)
because balances produced in the physics are required at the end of the
time step (e.g., the saturation adjustment in the microphysics).  So while
moving these non-RK3 physics into step (1) would eliminate the noise,
the balances produced by these physics would be altered.

We have found that the excitation of the acoustic modes can be
circumvented while leaving the non-RK3 physics in step (9) by using the
following procedure that is implemented in the ARW.  In step (1) of the
integration procedure (Fig. \ref{time_integration_figure}), an
estimate of the diabatic forcing in the $\Theta_m$ equation arising from the non-RK3
physics in step (9) is included in the diabatic forcing term
$R_{\Theta_m}^{t^*}$ in \eqref{theta-small-step} (which is advanced in step
4).  This estimated diabatic forcing is then removed from the updated
$\Theta_m$ after the RK3 integration is complete and before the evaluation
of the non-RK3 physics in step (9).  We use the diabatic forcing from
the previous time step as the estimated forcing; hence this procedure
results in few additional computations outside of saving the diabatic
forcing between time steps.

\subsection{Hydrostatic Option}

A hydrostatic option is available in the ARW solver.  The time-split RK3
integration technique summarized  in Fig. \ref{time_integration_figure}
is retained, including the acoustic
step loop.  Steps (5) and (6) in the acoustic-step loop,  where
$W$ and $\phi$ are advanced and $p''$ and $\alpha''$ are diagnosed, are 
replaced by the following three steps.  
(1) Diagnose the pressure from the full hydrostatic equation (including moisture)
%
\begin{equation}
\delta_\eta p_h = {\alpha_d \over \alpha} \mu_d = \bigl(1 + \sum q_m \bigr)\mu_d.
\notag
\end{equation}
%
\noindent
(2) Diagnose $\alpha_d$ using
the equation of state \eqref{ideal_gas_law}
and the prognosed $\theta_m$.
(3) Diagnose
the geopotential using the dry hydrostatic equation \eqref{hydro-pert}.
The vertical velocity
$w$ can be diagnosed from the geopotential equation, but it is not needed
in the solution procedure.  The acoustic step loop advances gravity waves,
including the external mode, and the Lamb wave 
when the hydrostatic option is used.

\section{Spatial Discretization}

The spatial discretization in the ARW solver uses a C grid staggering for
the variables as shown in Fig. \ref{figure:2}.  That is, normal
velocities are staggered one-half grid length from the thermodynamic
variables.  The variable indices, $(i,j,k)$ 
indicate variable locations with
$(x,y,\eta) = (i\Delta x, j\Delta y, k \Delta \eta)$.  We will denote
the points where $\theta$ is located as being {\it mass} points, and
likewise we will denote locations where $u$, $v$, and $w$ are defined as
$u$ points, $v$ points, and $w$ points, respectively.  Not shown in Fig.
\ref{figure:2} are the moisture 
variables $q_i$, and the coordinate metric $\mu$, defined at the
mass points on the discrete grid, and the geopotential $\phi$ that is
defined at the $w$ points.  The diagnostic variables used in the model, the
pressure $p$ and inverse density $\alpha$, are computed at mass points.
The grid lengths $\Delta x$ and $\Delta y$ are constants in the model
formulation; changes in the physical grid lengths associated with the
various projections to the sphere are accounted for using the map
factors introduced in Section \ref{spherical_projections}.  The vertical
grid length $\Delta \eta$ is not a fixed constant; it is specified in
the initialization.  The user is free to specify the $\eta$ values of the
model levels subject to the constraint that $\eta = 1$ at the surface,
$\eta = 0$ at the model top, and $\eta$ decreases monotonically between
the surface and model top.  Using these grid and variable definitions,
we can define the spatial discretization for the ARW solver.

%
% Figure 3.something
%
\begin{figure}
  \includegraphics *[width=6.5in,bb= 0 0 629.8 325.4]{figures/grids.pdf}
  \caption{\label{figure:2}Horizontal and vertical grids of the ARW}
\end{figure}

\subsection{Acoustic Step Equations}
\label{acoustic_discretization}

We begin by defining the column-mass-coupled variables relative to the
uncoupled variables.  The vertical velocity is staggered only in $k$, so
it can be coupled directly to the column mass with no averaging or
interpolation.  The horizontal velocities are horizontally staggered
relative to the column mass such that the continuous variables are 
represented discretely as
%
\begin{equation}
U = {\mu_d u \over m_y} \to 
{\overline{\mu_d}^x u \over {\overline m_y}^x} , 
~~~ V = {\mu_d v \over m_x} \to {\overline{\mu_d}^y v \over {\overline m_x}^y} , 
\notag
\end{equation}
%
\noindent
where the discrete operator 
$\overline{a}^x $
denotes a linear
interpolation operator.  The grid lengths $\Delta x$ and $\Delta y$ are
constant, hence in this case the operator reduces to $\overline{a}^x =
(a_{i+1/2} + a_{i-1/2})/2$.  

Using these definitions, we can write the spatially discrete 
acoustic step equations \hfill\break
\eqref{u-small-step} -- \eqref{geo-small-step} as
%
\begin{align}
\partial_t U'' + (m_x/m_y)\overline{(\alpha^{t^*}/\alpha^{t^*}_d)}^x \bigg[\overline{\mu_d^{t^*}}^x\bigg(
\overline{\alpha_d^{t^*}}^x  \partial_x {p''}^\tau  + \overline{{\alpha_d''}^\tau}^x  \partial_x {\overline p}
+ \partial_x \overline{{\phi''}^\tau}^\eta \bigg)
\notag &
\\
+ \partial_x \overline{\phi^{t^*}}^\eta \bigg(\partial_\eta \overline{\overline{{p''}}^x}^\eta - \overline{{\mu''_d}}^x \bigg)^\tau \,\bigg]  & = R_U^{t^*}
%%\delta_{\tau} U''  +  \overline{\mu^{t^*}}^x
%%   \overline{\alpha^{t^*}}^x \delta_x {p''}^\tau 
%% + (\overline{\mu^{t^*}}^x \delta_x \bar p)\overline{{\alpha''}^\tau}^x
%%~~~~~~~~~~~~~~~~~~~~~~~~~~~~~
%%\notag \\
%%  + \overline{(\alpha/\alpha_d)}^x 
%%    [\overline{\mu^{t^*}}^x \delta_x \overline{{\phi''}^\tau}^\eta
%% + (\delta_x \overline{\phi^{t^*}}^\eta)
%% (\delta_\eta \overline{\overline{p''}^x}^\eta 
%%    -\overline{\mu''}^x)^\tau] &= R_U^{t^*} 
\label{u-discrete}
\\
%
\partial_t V'' + (m_y/m_x)\overline{(\alpha^{t^*}/\alpha^{t^*}_d)}^y \bigg[\overline{\mu_d^{t^*}}^y\bigg(
\overline{\alpha_d^{t^*}}^y  \partial_y  {p''}^\tau  + \overline{{\alpha_d''}^\tau}^y  \partial_y {\overline p}
+ \partial_y \overline{{\phi''}^\tau}^\eta \bigg) 
 \notag &
\\
+ \partial_y \overline{\phi^{t^*}}^\eta \bigg(\partial_\eta \overline{\overline{{p''}}^y}^\eta - \overline{{\mu''_d}}^y \bigg)^\tau \, \bigg]  & = R_V^{t^*}
%%\delta_{\tau} V''  +  \overline{\mu^{t^*}}^y
%%   \overline{\alpha^{t^*}}^y \delta_y {p''}^\tau 
%% + (\overline{\mu^{t^*}}^y \delta_y \bar p)\overline{{\alpha''}^\tau}^y
%%~~~~~~~~~~~~~~~~~~~~~~~~~~~~~
%%\notag \\
%% + \overline{(\alpha/\alpha_d)}^y 
%%    [\overline{\mu^{t^*}}^y \delta_y \overline{{\phi''}^\tau}^\eta
%% + (\delta_y \overline{\phi^{t^*}}^\eta)
%% (\delta_\eta \overline{\overline{p''}^y}^\eta 
%%    -\overline{\mu''}^y)^\tau] &= R_V^{t^*}
\label{v-discrete}
\\
 \delta_\tau \mu_d'' 
 + m_x m_y[\delta_x U'' + \delta_y V'']^{\tau+\Delta \tau} 
+ m_y {\delta_\eta \Omega''^{\tau+\Delta \tau}} & = 
R_\mu^{t^*}
\label{mu-discrete}
\\
 \delta_\tau \Theta_m'' 
+ m_x m_y[  \delta_x (U''\overline{\theta_m^{t^*}}^x)
      + \delta_y (V''\overline{\theta_m^{t^*}}^y)]^{\tau+\Delta \tau} 
+ m_y \delta_\eta (\Omega''^{\tau+\Delta \tau} \overline{\theta_m^{t^*}}^\eta)
                                                &= {R_{\Theta_m}}^{t^*}  
\label{theta-discrete}
\\
 \delta_\tau W'' 
- m_y^{-1} g\overline{\left\{\overline{(\alpha/\alpha_d)^{t^*}}^\eta 
  \left[ \delta_\eta (C \delta_\eta \phi'') 
   +
\delta_\eta\biggl({c_s^2\over\alpha^{t^*}}{\Theta_m''\over\Theta_m^{t^*}}\biggr)
\right]
- \mu_d''\right\}}^\tau
 &= {R_W}^{t^*}  
\label{w-discrete}
\\
 \delta_\tau \phi'' + {1\over\mu_d^{t^*}}
[m_y \Omega''^{\tau+\Delta \tau} \delta_\eta \overline{\phi^{t^*}}^\eta  - m_y\overline{g W''}^\tau ] 
&= {R_\phi}^{t^*},
\label{geo-discrete}
\end{align}
%
\noindent
where the discrete operator 
%
\begin{equation}
\delta_x a = \Delta x^{-1} (a_{i+1/2} -
a_{i-1/2}) 
\end{equation}
%
\noindent
with the operators $\delta_y$ and $\delta_\eta$ similarily defined.
Additionally,
the operator $\overline{a}^\eta$ is a vertical interpolation operator.
Using the notation given for the vertically stretched grid depicted 
in Fig. \ref{figure:2}, it is defined as
%
\begin{equation}
\overline{a}^\eta|_{k+1/2} = {1 \over 2} \biggl(
{\Delta\eta_{k} \over \Delta\eta_{k+1/2}} a_{k+1} +
{\Delta\eta_{k+1} \over \Delta\eta_{k+1/2}} a_{k} \biggr).
\end{equation}
%
\noindent
This operator vertically interpolates variables on mass levels $k$ to the
$w$ levels $(k+{1\over 2})$.  It should be noted that the vertical grid
is defined such that vertical interpolation from $w$ levels to mass
levels reduces to 
$\overline{a}^\eta_k = (a_{k+1/2}+a_{k-1/2})/2$ (see Fig. \ref{figure:2}).

The RHS terms in the discrete acoustic step equations 
for momentum
\eqref{u-discrete}, \eqref{v-discrete} and \eqref{w-discrete}
are discretized as
%
\begin{align}
R_U^{t^*} =
 -(m_x/m_y) \overline{(\alpha/\alpha_d)}^x & \left[ \overline{\mu_d}^x (\partial_x \overline{\phi'}^\eta
+ \overline{\alpha_d}^x \partial_x p' + \overline{\alpha'_d}^x \partial_x \overline{p}) +
\partial_x \overline{\phi}^\eta (\partial_\eta \overline{\overline{p'}^x}^\eta - \overline{\mu'_d}^x)\right] & \cr
& + F_{U_{cor}} + \hbox{advection} + \hbox{mixing} + \hbox{physics,}
%%R_U^{t^*} = &- ( \overline{{\mu}_d}^x \overline{\alpha}^x \delta_x p' 
%%   -\overline{{\mu}_d}^x \overline{\alpha'}^x \delta_x \bar{p}) 
%%-\overline{(\alpha / \alpha_d)}^x
%%( \overline{{\mu}_d}^x \delta_x \overline{\phi'}^\eta
%%  -  \delta_\eta \overline{\overline{p'}^x}^\eta \delta_x \overline{\phi}^\eta
%%  + \overline{{\mu}_d'}^x \delta_x \overline{\phi}^\eta ) \notag  \\
%%& ~~~~~~~~~~~~~~~~~~~~~~~~~ + F_{U_{cor}} + \hbox{advection} +
%%\hbox{mixing} + \hbox{physics},
\label{u-pg-discrete} \\
%
R_V^{t^*} =
 -(m_y/m_x) \overline{(\alpha/\alpha_d)}^y & \left[ \overline{\mu_d}^y (\partial_y \overline{\phi'}^\eta
+ \overline{\alpha_d}^y \partial_y p' + \overline{\alpha'_d}^y \partial_y \overline{p}) +
\partial_y \overline{\phi}^\eta (\partial_\eta \overline{\overline{p'}^y}^\eta - \overline{\mu'_d}^y)\right] & \cr
& + F_{V_{cor}} + \hbox{advection} + \hbox{mixing} + \hbox{physics,}
%%R_V^{t^*} = &- ( \overline{{\mu}_d}^y \overline{\alpha}^y \delta_y p' 
%%   -\overline{{\mu}_d}^y \overline{\alpha'}^y \delta_y \bar{p}) 
%%-\overline{(\alpha / \alpha_d)}^y
%%( \overline{{\mu}_d}^y \delta_y \overline{\phi'}^\eta
%%  -  \delta_\eta \overline{\overline{p'}^y}^\eta \delta_y \overline{\phi}^\eta
%%  + \overline{{\mu}_d'}^y \delta_y \overline{\phi}^\eta ) \notag  \\
%%& ~~~~~~~~~~~~~~~~~~~~~~~~~ + F_{V_{cor}} + \hbox{advection} +
%%\hbox{mixing} + \hbox{physics},
\label{v-pg-discrete} \\
%
R_W^{t^*} = ~ m_y^{-1} g \overline{(\alpha/\alpha_d)}^\eta [\delta_\eta p' 
+ & {\bar{\mu}}_d \overline{q_m}^\eta]
- m_y^{-1} {\mu}_d'g & \cr
& + F_{W_{cor}} + \hbox{advection} + \hbox{mixing} + \hbox{buoyancy} + \hbox{physics}.
\label{w-pg-discrete}
\end{align}

\subsection{Coriolis and Curvature Terms}

The terms $F_{U_{cor}}$, $F_{V_{cor}}$, and $F_{W_{cor}}$ in 
\eqref{u-pg-discrete} -- \eqref{w-pg-discrete}
represent Coriolis and
curvature effects in the momentum equations and are written in continuous form in
\eqref{u-mom-rhs} -- \eqref{w-mom-rhs}. Using the isotropic map projections 
(Lambert conformal, polar stereographic, and Mercator) where $m_x=m_y=m$,
 their spatial discretization is
%
\begin{align}
F_{U_{cor}} & =  + 
\bigl({\overline f}^x +
{\overline{
  {\overline u}^x 
   {\delta_y m} 
- {\overline v}^y 
    \delta_x m}}^x     \bigr)
{\overline V}^{xy}
- {\overline e}^x
{\overline W}^{x\eta}\, {\overline{\cos \alpha_r}}^x 
- {u {\overline W}^{x\eta} \over r_e},
\label{coriolis-u-regional}
\\
%
F_{V_{cor}} & =  - 
\bigl({\overline f}^y +
{\overline{
{\overline u}^x 
{\delta_y m} 
- {\overline v}^y \delta_x m}}^y \bigr)
{\overline U}^{xy}
+ {\overline e}^y
{\overline W}^{y\eta}\, {\overline{\sin \alpha_r}}^y 
- {v {\overline W}^{y\eta} \over r_e},
\label{coriolis-v-regional}
\\
%
F_{W_{cor}} & = + e ({\overline U}^{x\eta}
\cos \alpha_r - {\overline V}^{y\eta}
\sin \alpha_r) + \biggl({ 
 {\overline u}^{x\eta}  {\overline U}^{x\eta} 
+{\overline v}^{y\eta}  {\overline V}^{y\eta} 
\over r_e}\biggr).
\label{coriolis-w-regional}
\end{align}
%
\noindent
Here the operators $\overline{()}^{xy} = \overline{\overline{()}^{x}}^y
$, and likewise for $\overline{()}^{x\eta}$ and $\overline{()}^{y\eta}$.

For the non-isotropic latitude longitude projection, the Coriolis and
curvature terms are discretized as
%
\begin{align}
F_{U_{cor}} & =  {m_x \over m_y} \biggl[\hphantom{-}{\overline f}^x 
{\overline V}^{xy}
+ {u\overline{V}^{xy} \over r_e} \tan\psi \biggr] 
- {\overline e}^x
{\overline W}^{x\eta}\, {\overline{\cos \alpha_r}}^x 
- {u {\overline W}^{x\eta} \over r_e},
\label{coriolis-u-global}
\\
%
F_{V_{cor}} & =  {m_y \over m_x} \biggl[ - \overline{f}^y
\overline{U}^{xy} 
- {\overline{u}^{xy} \overline{U}^{xy} \over r_e} \tan \psi  
+ \overline{e}^y \overline{W}^{y\eta}\overline{\sin \alpha_r}^y
%% + {v\overline{W}^{y\eta} \over r_e}
- {v\overline{W}^{y\eta} \over r_e}
\biggr],
\label{coriolis-v-global}
\\
%
F_{W_{cor}} & = + e ({\overline U}^{x\eta}
\cos \alpha_r - (m_x/m_y) {\overline V}^{y\eta}
\sin \alpha_r) + \biggl({ 
 {\overline u}^{x\eta}  {\overline U}^{x\eta} 
+(m_x/m_y) {\overline v}^{y\eta}  {\overline V}^{y\eta} 
\over r_e}\biggr).
\label{coriolis-w-global}
%
\end{align}



\subsection{Advection}
\label{advection}

The advection terms in the ARW solver are in the form of a flux divergence and
are a subset of the RHS terms in equations \eqref{u-rhs} --
\eqref{geo-rhs}:
%
\begin{align}
R_{U_{adv}}^{t^*} = &
- m_x[\partial_x(Uu) + \partial_y(Vu)] + \hphantom{(m_x/m_y)} \partial_\eta (\Omega u)
\\
%
R_{V_{adv}}^{t^*} = &
- m_y[\partial_x (Uv) + \partial_y (Vv)] + (m_x/m_y) \partial_\eta (\Omega v)
\\ 
%
R_{\mu_{adv}}^{t^*} = &
%% - m_x m_y[U_x + V_y] + m_y \Omega_\eta
- m_x m_y[\partial_x U + \partial_y V] + m_y \partial_\eta\Omega
\\
R_{\Theta_{adv}}^{t^*} = &
- m_x m_y [\partial_x (U\theta_m) + \partial_y (V\theta_m)] - m_y \partial_\eta
(\Omega \theta_m) 
\\
%
R_{W_{adv}}^{t^*} = &
- m_x [\partial_x (Uw) + \partial_y (Vw)] + \partial_\eta
(\Omega w) 
\\
%
R_{\phi_{adv}}^{t^*} = &
- \mu_d^{-1}
%% [m_x m_y (U\phi_x + V\phi_y) + m_y
%% \Omega\phi_\eta].
[m_x m_y (U\partial_x \phi + V\partial_y \phi) + m_y
\Omega\partial_\eta\phi].
\end{align}
%
\noindent
For the mass conservation equation, the flux divergence is discretized
using a 2nd-order centered approximation:
%
\begin{equation}
R_{\mu_{adv}}^{t^*} = 
- m_x m_y [\delta_x U + \delta_y V]^{t^*} + m_y \delta_\eta \Omega^{t^*}.
\end{equation}
%

In the current version of the ARW, the advection of vector quantities
(momentum) and scalars is performed
using the RK3 time integration as outlined in Fig. 
\ref{time_integration_figure}.  The spatial discretization used in this
approach is outlined in the next section.  For many applications it
is desirable to use positive definite or monotonic advection schemes
for scalar transport.  

%In the next major release of the ARW we will be
%including a forward-in-time scheme for scalar transport that has
%positive definite and monotonic options.  We describe that scheme in
%the section following the description of the RK3 advection.

\subsubsection{RK3 Advection}
\label{rk3-advection}

$\hbox{2}^{nd}$ through $\hbox{6}^{th}$ order accurate spatial 
discretizations
of the flux divergence are available in the ARW for momentum,
scalars and geopotential using the RK3 time-integration scheme
(scalar advection option 1, step 7 in the time-split integration sequence
in Fig. \ref{time_integration_figure}).
The discrete operators can be illustrated
by considering the flux divergence equation for
a scalar $q$ in its discrete form:
%
\begin{equation}
R_{q_{adv}}^{t^*} =
- m_x m_y [\delta_x (U \overline{q}^{x_{adv}}) 
+ \delta_y (V\overline{q}^{y_{adv}})] 
- m_y \delta_\eta
(\Omega \overline{q}^{\eta_{adv}}).
\label{flux-divergence}
\end{equation}
%
As in the pressure gradient discretization, the discrete operator
is defined as
%
\begin{equation}
\delta_x (U \overline{q}^{x_{adv}}) = \Delta x^{-1} 
\bigl[ (U \overline{q}^{x_{adv}})_{i+1/2} -
(U \overline{q}^{x_{adv}})_{i-1/2} \bigr].
\label{discrete-divergence}
\end{equation}
%
\noindent
The different order advection schemes correspond to different
definitions for the operator  $\overline{q}^{x_{adv}}$.  
The even order operators 
($\hbox{2}^{nd}$, $\hbox{4}^{th}$, and $\hbox{6}^{th}$) are
%
\begin{align}
\hbox{2}^{nd} \hbox{ order:} ~~~~ & 
(\overline{q}^{x_{adv}})_{i-1/2} = 
{1 \over 2} 
(q_{i} + q_{i-1}) \notag \\
%
\hbox{4}^{th} \hbox{ order:} ~~~~ & 
(\overline{q}^{x_{adv}})_{i-1/2} = 
{7 \over 12} (q_{i} + q_{i-1}) 
- {1 \over 12} (q_{i+1} + q_{i-2}) 
\notag \\
\hbox{6}^{th} \hbox{ order:} ~~~~ & 
(\overline{q}^{x_{adv}})_{i-1/2} = 
{37 \over 60} (q_{i} + q_{i-1}) 
- {2 \over 15} (q_{i+1} + q_{i-2}) 
+ {1 \over 60} (q_{i+2} + q_{i-3}),
\notag
\end{align}
%
\noindent
and the odd order operators ($\hbox{3}^{rd}$ and $\hbox{5}^{th}$) are
%
\begin{align}
\hbox{3}^{rd} \hbox{ order:} ~~~~ & 
(\overline{q}^{x_{adv}})_{i-1/2} = 
(\overline{q}^{x_{adv}})_{i-1/2}^{4^{th}} 
\notag
\\
&~~~~~~~~~~~~~~~~~~~~
+ \hbox{sign}(U){1 \over 12} \bigl[
(q_{i+1}-q_{i-2}) - 3(q_i - q_{i-1}) \bigr]
\notag
\\
\hbox{5}^{th} \hbox{ order:} ~~~~ & 
(\overline{q}^{x_{adv}})_{i-1/2} = 
(\overline{q}^{x_{adv}})_{i-1/2}^{6^{th}} 
\notag
\\
&~~~~~~~~~~~~~~~~~~~~
- \hbox{sign}(U){1 \over 60} \bigl[
(q_{i+2}-q_{i-3}) - 5(q_{i+1} - q_{i-2}) 
+ 10(q_i - q_{i-1})
\bigr].
\notag
\end{align}

The even-order advection operators are spatially centered and thus
contain no implicit diffusion outside of the diffusion inherent in
the RK3 time integration.  The odd-order schemes are upwind-biased, and
the spatial discretization is inherently diffusive.  The behavior of 
the upwind schemes is easily understood by expanding
\eqref{discrete-divergence} using the $5^{th}$ order operator, assuming a
constant mass flux $U$ and
multiplying by the timestep $\Delta t$:
%
\begin{align}
\Delta t \delta_x (U \overline{q}^{x_{adv}}) =  
& \, {\Delta t} {\delta (Uq)|}^{6th}
-\left| {U \Delta t \over \Delta x }\right| {1 \over 60}
\left(-q_{i-3}+6q_{i-2}-15q_{i-1}+20q_{i}-15q_{i+1}+6q_{i+2}-q_{i+3}
\right)
\notag
\\
= & \, {\Delta t} {\delta (Uq)|}^{6th} - {Cr \over 60} \Delta x^6 {\partial^6
%% q \over \partial x^6} + \, H.O.T.
q \over \partial x^6} + \, {\rm higher\ order\ terms}
\notag
\end{align}
%
\noindent
Similarly, we can expand \eqref{discrete-divergence} using the $3^{rd}$
order operator:
\begin{align}
\Delta t \delta_x (U \overline{q}^{x_{adv}})
= & \, {\Delta t} {\delta (Uq)|}^{4th} + {Cr \over 12} \Delta x^4 {\partial^4
%% q \over \partial x^4} + \, H.O.T.
q \over \partial x^4} + \, {\rm higher\ order\ terms}
\notag
\end{align}
%
\noindent
As is evident in
their formulation, the odd-order schemes are comprised of the next
higher (even) order centered scheme plus an upwind term that, for a
constant transport mass flux, is a diffusion term of that next higher
(even) order with a hyper-viscosity proportional to the Courant number
($Cr$).
The diffusion term is the leading order error term in the flux
divergence discretization.
Further details concerning RK3 advection can be found in \citet{wicker02}

\subsubsection{Positive-Definite and Shape-Preserving Limiters for RK3 Advection}
\label{positive-definite-transport}

Mixing ratios of moisture, chemical species or other tracer species
should remain positive-definite, that is, negative masses should not be
permitted.  Additionally, transport should not introduce any new maxima or minima 
in the mixing ratios; this property is referred to as shape-preservation also monotonicity.
The Runge-Kutta transport integration defined by the
timestepping algorithm \eqref{rk3a} -- \eqref{rk3c}, combined with the
flux divergence operator \eqref{flux-divergence}, is conservative but it
does not guarantee positive definiteness or shape preservation; any negative values will be
offset by positive mass such that mass is conserved, and new maxima and minima may be produced.  In many physics
options, negative mixing ratios will be set to zero, and this will
result in an increase in mass of that species.  Both positive-definite and shape-preserving flux
renormalizations, applied on the final Runge-Kutta transport step
\eqref{rk3a}, can be used to remove these unphysical effect from the RK3
scalar transport scheme.  The positive-definite renormalization in the ARW solver is described in 
\citet{skamarock-weisman-08} and the shape-preserving (monotonic) extension is described in \citet{Wang-et-al-2009}, 
both of which follow an approach first described by \citet{Zalesak-1979}.

The renormalizations occur as part of the final RK3 time integration step for scalar transport, and the preliminary
(i.e. before renormalization) evaluation of the final transport step can be
expressed as
%
\begin{equation}
(\mu\phi)^{t+\Delta t} = (\mu\phi)^{t} - {\Delta t}\bigl\{
m_x m_y [\delta_x (U \overline{q}^{x_{adv}}) 
+ \delta_y (V\overline{q}^{y_{adv}})] 
+ m_y \delta_\eta
(\Omega \overline{q}^{\eta_{adv}})\bigr\}
+ \Delta t \mu S_\phi^t
\label{rk3c-scalar}
\end{equation}
%
\noindent
where the flux divergence is evaluated using the (**) time level
predicted in RK3 step \eqref{rk3b}.  The positive-definite flux
renormalization replaces \eqref{rk3b} with the following two steps.  First,
the scalar mixing ratio is updated using the tendency derived from the
model physics and source/sink terms.
%
\begin{equation}
(\mu\phi)^{***} = (\mu\phi)^{t} + {\Delta t} \mu S_\phi^t
\label{scalar-source-update}
\end{equation}
%
where we denote this new predictor as $(\mu\phi)^{***}$.  Second, the
full update is computed using a
flux divergence composed of a first-order upwind flux plus a
higher order correction:

\begin{align}
(\mu\phi)^{t+\Delta t} = (\mu\phi)^{***} - {\Delta t}\bigl\{ \,\,\,
m_x m_y ( \delta_x &\left[ (U \overline{q}^{x_{adv}})^1 
+ R(U \overline{q}^{x_{adv}})' \right] 
\notag
\\
+\delta_y &\left[(V\overline{q}^{y_{adv}})^1 
+ R(V\overline{q}^{y_{adv}})'\right] )
\notag
\\
+ m_y \delta_\eta &\left[
(\Omega \overline{q}^{\eta_{adv}})^1 + R(\Omega \overline{q}^{\eta_{adv}})' \right]
\,\,\,\, \bigr\}.
\label{pd-update}
\end{align}
%
In \eqref{pd-update}, $()^1$ denotes a first-order upwind flux and $R()'$
denotes a renormalized higher-order correction flux.  The higher-order
correction flux is the difference between the full RK3 flux and the
first-order upwind flux, that is,
%
\begin{equation}
(U \overline{q}^{x_{adv}}) = (U \overline{q}^{x_{adv}})^1 + (U
\overline{q}^{x_{adv}})',
\label{correction-def}
\end{equation}
%
with similar definitions for $(V \overline{q}^{y_{adv}})'$ 
and $(\Omega \overline{q}^{\eta_{adv}})'$.  The correction flux can then
renormalized to achieve positive-definite or shape preserving (monotonic) behavior as follows.  
First, the upwind fluxes are used to perform a
partial update of the scalar mass.
%
\begin{equation}
(\tilde{\mu\phi}) = (\mu\phi)^{***} 
- {\Delta t} \bigl\{
m_x m_y [\delta_x (U \overline{q}^{x_{adv}})^1 
+ \delta_y (V\overline{q}^{y_{adv}})^1] 
+ m_y \delta_\eta
(\Omega \overline{q}^{\eta_{adv}})^1\bigr\}
\notag
\end{equation}
%
This update is positive definite (and monotonic) by design because it is a property of the
the first-order upwind scheme.  Next, a prediction of the minimum and maximum
possible values of the new-time-level species mass is computed at each
point by using only the outward directed fluxes (fluxes that remove mass
from the control volume),
%
\begin{equation}
(\mu\phi)_{min}^{t+\Delta t} = (\tilde{\mu\phi}) 
-{\Delta t} \bigl\{
m_x m_y [\delta_x (U_+ \overline{q}^{x_{adv}})' 
+ \delta_y (V_+\overline{q}^{y_{adv}})'] 
+ m_y \delta_\eta
(\Omega_+ \overline{q}^{\eta_{adv}})'\bigr\}
\label{min-prediction}
\end{equation}
%
\begin{equation}
(\mu\phi)_{max}^{t+\Delta t} = (\tilde{\mu\phi}) 
-{\Delta t} \bigl\{
m_x m_y [\delta_x (U_- \overline{q}^{x_{adv}})' 
+ \delta_y (V_-\overline{q}^{y_{adv}})'] 
+ m_y \delta_\eta
(\Omega_- \overline{q}^{\eta_{adv}})'\bigr\}
\label{max-prediction}
\end{equation}
%
where $U_+$, $V_+$ and $\Omega_+$ indicated the use of fluxes 
out of the control volume only, that is, only those that contribute to
lowering the scalar mass,
and $U_-$, $V_-$ and $\Omega_-$ indicated the use of fluxes 
into the control volume only, that is, only those that contribute to
rasing the scalar mass.  

For the positive-definite renormalization, from \eqref{min-prediction}
the scalar mass $(\mu\phi)_{min}^{t+\Delta t} < 0$ if and only if
%
\begin{equation}
(\tilde{\mu\phi}) < 
{\Delta t} \bigl\{
m_x m_y [\delta_x (U_+ \overline{q}^{x_{adv}})' 
+ \delta_y (V_+\overline{q}^{y_{adv}})'] 
+ m_y \delta_\eta
(\Omega_+ \overline{q}^{\eta_{adv}})'\bigr\}.
\notag
\end{equation}
%
For each volume where negative mass is indicated by
\eqref{min-prediction}, the fluxes are renormalized
such that the outgoing fluxes and mass in the volume are equivalent.
%
\begin{equation}
R\left(U_+\overline{q}^{x_{adv}}\right)' = 
\left(U_+\overline{q}^{x_{adv}}\right)' 
{ \tilde{\mu \phi} \over
{\Delta t} \bigl\{
m_x m_y [\delta_x (U_+ \overline{q}^{x_{adv}})' 
+ \delta_y (V_+\overline{q}^{y_{adv}})'] 
+ m_y \delta_\eta
(\Omega_+ \overline{q}^{\eta_{adv}})'\bigr\} }
\notag %\label{renormalization}
\end{equation}
%
with a similar renormalization applied to the 
$(V_+ \overline{q}^{y_{adv}})'$ 
and $(\Omega_+ \overline{q}^{\eta_{adv}})'$.


For the shape-preserving (monotonic) renormalization, the minimum (maximum) value
for a cell mixing ratio at the new time level is set as the minimum (maximum) mixing ratio at time $t$ from that 
cell and its nearest neighbors.  This lower bound on the mixing ratio, 
denoted $(\phi)_{min}$, will be violated when
%
\begin{equation}
(\tilde{\mu\phi}) - (\mu \phi)_{min} < 
{\Delta t} \bigl\{
m_x m_y [\delta_x (U_+ \overline{q}^{x_{adv}})' 
+ \delta_y (V_+\overline{q}^{y_{adv}})'] 
+ m_y \delta_\eta
(\Omega_+ \overline{q}^{\eta_{adv}})'\bigr\}
\notag
\end{equation}
%
If the lower bound is violated,
the fluxes are renormalized with
%
\begin{equation}
R\left(U_+\overline{q}^{x_{adv}}\right)' = 
\left(U_+\overline{q}^{x_{adv}}\right)' 
{ \tilde{\mu \phi} - (\mu \phi)_{min}\over
{\Delta t} \bigl\{
m_x m_y [\delta_x (U_+ \overline{q}^{x_{adv}})' 
+ \delta_y (V_+\overline{q}^{y_{adv}})'] 
+ m_y \delta_\eta
(\Omega_+ \overline{q}^{\eta_{adv}})'\bigr\} }
\notag %\label{renormalization}
\end{equation}
%
with a similar renormalization applied to the 
$(V_+ \overline{q}^{y_{adv}})'$ 
and $(\Omega_+ \overline{q}^{\eta_{adv}})'$.
A similar limiting process is used to renormalize the fluxes such that the maximum mixing ratio is not exceeded.  
Note that outgoing fluxes for one cell are inflow fluxes for another, and the renormalized flux used in the update is the minimum flux necessary to satisfy the mimimum or maximum bounds for the two cells utilizing the flux.

The renormalized higher-order-correction fluxes along
with the first order fluxes are then used in the update equation
\eqref{pd-update}.  Note that if no renormalization is needed the
scheme \eqref{pd-update} reverts to the standard RK3 update because of
the definition of the higher-order correction \eqref{correction-def}.


\subsection{Pole Conditions for the Global Latitude-Longitude Grid}
\label{pole_condition}

The latitude-longitude grid has a singularity at the two poles where the
latitude $\psi = \pm 90^\circ$, as illustrated in Figure
\ref{figure_pole}.  By design, no variable is defined at the pole point.
The area of the control volume face at the pole is zero, thus a flux at
the pole point is not needed in the solution of any of the prognostic
variables.  For example, in the finite volume discretization of the mass
conservation equation \eqref{mu-discrete} for $\mu$ in a control volume
closest to the pole, the meridional gradient ($y$ gradient) of the mass
flux $\delta_y V''$ will use zero for the pole contribution to this
term.

%
%  figure 3.2 for polar b.c
%
\begin{figure}
  \includegraphics *[width=6.5in,bb= 0 0 419 183]{figures/polar_bc.pdf}
  \caption{\label{figure_pole} Latitude-longitude grid structure in the
pole region. In the ARW formulation, $r_e\Delta\psi = \Delta y/m_y $.}
\end{figure}

The stencils for advection operators higher than 2nd order, described in
Section \ref{rk3-advection}, cross the poles for flux calculations at
the control volume faces located $r_e\Delta\psi$ and $2r_e\Delta\psi$
from the pole point (Figure \ref{figure_pole}, the $V$ flux is indicated
in red).  In the current implementation of the ARW, we reduce the order
of the flux operator at these faces so that their stencils do not extend
across the pole point.  While this formally reduces the accuracy of the
scheme, we have not been able to identify any significant degradation in
the ARW solutions.

Coriolis and curvature terms are computed for the vertical momentum
equation \eqref{coriolis-w-global} and the horizontal momentum equation
for $U$ \eqref{coriolis-u-global}.  For the $W$ and $U$ points that lie
$r_e \Delta \psi/2$ from the pole, the stencils for these terms require a
value of $V$ at the pole point.  We set the value of $V$ at the pole
equal to the value of $V$ at $\Delta \psi$ to evaluate these
operators. This approximation is also used for the meridional advection
of $V$, combined with a lowering of the flux-operator order to avoid
differencing across the pole (as with the flux divergence terms for the
other prognostic variables).

\section{Stability Constraints}

There are two time steps that a user must specify when running the ARW:
the model time step (the time step used by the RK3 scheme, see 
Section \ref{rk3_scheme}) and the acoustic time step (used in the 
acoustic sub-steps of the time-split integration procedure,
see Section \ref{acoustic_equations}).  Both are limited by Courant
numbers. 
%Note that the thermal buoyancy terms are handled on the acoustic
%steps, and therefore gravity wave frequencies, which are lower than acoustic
%frequencies, do not limit the time step.
In the following sections we describe how to choose time steps
for applications.

\subsection{RK3 Time Step Constraint}
\label{rk3_timestep_constraint}

The RK3 time step is limited by the advective Courant number $u \Delta
t/\Delta x$ and the user's choice of advection schemes--- users can
choose $2^{nd}$ through $6^{th}$ order discretizations for the advection
terms.  The time-step limitations for 1D advection in the RK3 scheme 
using these advection schemes is given in \citet{wicker02}, 
and is reproduced here.

\begin{table}[h]
\begin{center}
\begin{tabular}{|c|c|c|c|c|} \hline
& \multicolumn{4}{c|}{Spatial order} \\\cline{2-5}
\raisebox{1.5ex}[0cm]{Time Scheme}
& 3rd & 4th & 5th & 6th \\\hline
Leapfrog & {\em Unstable} & 0.72 & {\em Unstable} & 0.62 \\\hline
RK2 & 0.88 & {\em Unstable} & 0.30 & {\em Unstable} \\\hline
RK3 & 1.61 & 1.26 & 1.42 & 1.08 \\\hline
\end{tabular}
\end{center}
\caption{Maximum stable Courant numbers for one-dimensional linear
advection.  From \citet{wicker02}.}
\label{courant_table}
\end{table}

As is indicated in the table, the maximum stable Courant numbers for
advection in the RK3 scheme are almost a factor of two greater than
those for the leapfrog time-integration scheme.  For advection in three
spatial dimensions, the maximum stable Courant number is $1/\sqrt{3}$
times the Courant numbers given in Table \ref{courant_table}.  For
stability, the time step used in the ARW should produce a maximum
Courant number less than that given by theory.  Thus, for 3D
applications, the time step should satisfy the following equation:
%
\begin{equation}
\Delta t_{max} < {Cr_{theory} \over \sqrt{3}} \cdot {\Delta x \over
u_{max}},
\label{rk3_timestep}
\end{equation}
%
\noindent
where $Cr_{theory}$ is the Courant number taken from the RK3 entry in
Table \ref{courant_table} and $u_{max}$ is the maximum velocity expected
in the simulation.  For example in real-data applications, where jet
stream winds may reach as high as $100\,\,\hbox{ms}^{-1}$, the maximum
time step would be approximately 80 s on a $\Delta x = 10 \,\,\hbox{km}$ grid 
using $5^{th}$ order advection.  For convection-permitting resolutions
(typically $\Delta x \le 5$ km), the vertical velocities in convective
updrafts produce the stability-limiting Courant numbers.
Given additional constraint from the time splitting, and to provide a
safety buffer, we usually choose a time step that is approximately 25\%
less than that given by \eqref{rk3_timestep}.  This time step is
typically a factor of two greater than that used in leapfrog-based
models.  For those users familiar with the MM5 model, the rule of thumb
for choosing a time step is that the time step, in seconds, should be 
approximately 3 times the horizontal grid distance, in kilometers.
For the ARW, the time step (in seconds) should be approximately 
6 times the grid distance (in kilometers).  

\subsection{Acoustic Time Step Constraint}
\label{acoustic_step_constraint}

The forward-backward time integration scheme used in the ARW's 2D
explicit acoustic step integration allows a maximum Courant number
$Cr_{max} = c_s \Delta \tau/\Delta x < 1/\sqrt{2}$, where $c_s$ is the
speed of sound.  We typically use a more conservative estimate for this
by replacing the limiting value $1/\sqrt{2}$ with $1/2$.  Thus, the acoustic
time step used in the model is
%
\begin{equation}
\Delta \tau < 2 \cdot {\Delta x \over c_s}.
\label{acoustic_timestep}
\end{equation}
%
\noindent
For example, on a $\Delta x = 10 \,\, \hbox{km}$ grid, using a sound
speed $c_s = 300 \,\, \hbox{ms}^{-1}$, the acoustic time step given in
\eqref{acoustic_timestep} is approximately 17 s.  In the ARW, the
ratio of the RK3 time step to the acoustic time step must be an even
integer.  For our example using a $\Delta x = 10$ km grid in a
real-data simulation, we would specify the RK3 time step $\Delta t = 60
s$ (i.e., 25\% less than the 80 s step given by \eqref{rk3_timestep},
and an acoustic time step $\Delta \tau = 15$ s (i.e., 1/4 of the RK3
step, rounding down the $\Delta \tau = 17$ s step given by
\eqref{acoustic_timestep}).  Note that it is the ratio of the RK3
time step to the acoustic time step that is the required input in the ARW.


\subsection{Adaptive Time Step}
\label{adaptive_time_step}

The ARW model is typically integrated with a fixed timestep, 
that is chosen to produce a stable integration.  During any time in
the integration, the maximum stable timestep is likely to be larger than
the fixed timestep.  In ARWV3, an adaptive timestepping capability has
been introduced that chooses the RK3 timestep based on the temporally-evolving
wind fields.  The adaptively-chosen timestep is usually larger than the
typical fixed timestep, hence the dynamics integrates faster and
physics are called less often, and the time-to-completion of the
simulation can be substantially reduced.

In the adaptive timestep scheme, a target maximum Courant number
$Cr_{target}$ is chosen, where typically $1.1 \leq Cr_{target} \leq
1.2$.  The maximum Courant number in the domain at a given time ($Cr_{domain}$), computed
for all the velocity components $(u,v,w)$, is then used to compute a new
timestep.  When the maximum Courant number in the domain is less than
the target maximum Courant number $(Cr_{domain} < Cr_{target})$, 
then the timestep can be increased
and the new timestep is computed using
%
\begin{equation}
\Delta t_{current} = min \Big( 1+f_{i} , 
{Cr_{target} \over Cr_{domain}}\Big) \cdot \Delta t_{previous},
\label{adapt_inc_timestep}
\end{equation}
%
\noindent
where a typical value for the regulated increase is $f_{i} \leq 5\%$.  When
the computed maximum domain-wide Courant number exceeds the targeted
maximum allowable Courant number $(Cr_{domain} > Cr_{target})$, 
then the time step is 
decreased to insure model stability:
%
\begin{equation}
\Delta t_{current} = max \Big( 1-f_{d} , 
{{Cr_{target} - {0.5 ({Cr_{domain} - Cr_{target}}) }} 
\over Cr_{domain} } \Big) \cdot \Delta t_{previous},
\label{adapt_dec_timestep}
\end{equation}
%
\noindent
where typically the factor to decrease the time step $f_{d} = 25\%$.  Both a lower bound and
an upper bound on the time step are enforced based on the initial settings 
of the time step suggested in section \ref{rk3_timestep_constraint}:
%
\begin{equation}
\Delta t_{init}  = 6 \cdot \Delta x ,
\label{adapt_init_timestep}
\end{equation}
\begin{equation}
\Delta t_{min} = 0.5 \cdot \Delta t_{init},
\label{adapt_bound_min_timestep}
\end{equation}
\begin{equation}
\Delta t_{max} = 3.0 \cdot \Delta t_{init},
\label{adapt_bound_max_timestep}
\end{equation}
\begin{equation}
\Delta t = min ( max ( \Delta t, \Delta t_{min} ) , \Delta t_{max} ),
\label{adapt_bound_min_timestep}
\end{equation}
%
\noindent
to guard the time step from becoming too small or too large.  
The computation of the number of acoustic time steps $n_s$, described in section
\ref{acoustic_step_constraint}, is handled during the first of each of the
RK3 model integration steps:
%
\begin{equation}
n_s
= max \Big( 2 \cdot \lfloor { 300 \cdot {\Delta t \over \Delta x} + 1 } \rfloor , 4 \Big).
\label{adapt_acoustic_timestep}
\end{equation}
%

For a simulation using nests, the fine-grid domain must maintain an integer number of 
time steps within a single model integration step from the parent.  At each of 
the starting
model integration steps for the child grid, those integration steps when the time on the fine-grid 
domain equals the time
on the parent's domain, the adaptive time step algorithm is conducted for the fine grid.
Note that the ratio of the nominal grid distance between the parent and the child does not
necesarily imply that same ratio between the model integration time steps.


\subsection{Map Projection Considerations}
\label{global_considerations}

For ARW configurations using the Lambert conformal, polar stereographic,
or Mercator projections, the timestep constraints is determined by
the smallest physical horizontal grid spacing,
i.e. $\hbox{min}(\Delta x/m_x, \Delta y/m_y)$.  For global applications,
the grid distance used to determine the timestep should be $\Delta
x/m_x$ evaluated at the computational latitude at which the polar
filters are activated.  Polar filtering is discussed in section
\ref{polar_filter_section}.
