\chapter{Four-Dimensional Data Assimilation}
\label{fdda_chap}

Four-dimensional data assimilation (FDDA), also known as nudging, is a method of keeping
simulations close to analyses and/or observations or forcing data over the course of an
integration by applying extra forcing terms to the model equations.
There are three types of FDDA and some can be used in
combination. Grid- or analysis-nudging simply forces the model simulation 
towards a series of analyses grid-point by grid-point. Observational- or station-nudging
locally forces the simulation towards observational data. 
Spectral nudging only forces the model to the selected spectra of waves in the analysis.
These methods provide a four-dimensional analysis that is somewhat balanced dynamically,
and in terms of continuity,
while allowing for complex local topographical or convective variations.
Such datasets can cover long periods, and have particular value in driving
off-line air quality or atmospheric chemistry models.

\section{Grid Nudging or Analysis Nudging}

Grid nudging is a component of FDDA. Other components are surface-analysis nudging, and observational nudging. The method can be used to nudge wind component u and v, temperature and water vapor mixing ratio.

The grid-nudging method is specifically three-dimensional analysis nudging, whereby the atmospheric model is nudged towards time- and space-interpolated analyses using a point-by-point relaxation term. \citet{stauffer90} originally developed the technique for MM5.

The grid-nudging technique has several major uses.

{\it a) Four-dimensional datasets.} The model is run with grid-nudging for long periods, e.g. months, to provide a four-dimensional meteorologically self-consistent dataset that also stays on track with the driving analyses. In this way, the model is used as an intelligent interpolator of analyses between times, and also accounting better for topographic and convective effects. As mentioned, the primary use for such datasets is in air quality where the wind fields may be used to drive off-line chemistry models.

{\it b) Boundary conditions.} A nested simulation is run with the outer domain nudged towards analyses, and the nest running un-nudged. This provides better temporal detail at the nest boundary than driving it directly from linearly interpolated analyses, as it would be if it were the outer domain. This technique could also be used in forecasting, where an outer domain is nudged towards global forecast fields that are available in advance of the regional forecast.

{\it c) Dynamic initialization.} A pre-forecast period (e.g., -6 hours to 0 hours) is run with nudging using analyses at those times that are already available. This is probably better than a cold-start using just the 0 hour analysis because it gives the model a chance to spin up. In particular, the model will have six hours to adjust to topography, and produce cloud fields by hour 0, whereas with a cold start there would be a spin-up phase where waves are produced and clouds are developed. This method could also be combined with 3D-Var techniques that may provide the hour 0 analysis.

Grid nudging has been added to ARW using the same input analyses as the WPS pre-processing systems can provide. Since it works on multiple domains in a nesting configuration, it requires multiple time-periods of each nudged domain as input analyses. Given these analyses, the {\it real} program produces another input file which is read by the model as nudging is performed. This file contains the gridded analysis 3d fields of the times bracketing the current model time as the forecast proceeds. The four nudged fields are the two horizontal wind components (u and v), temperature, and specific humidity. 

The method is implemented through an extra tendency term in the nudged variable's equations, e.g.

$$ {\partial \theta \over \partial t} = F(\theta) + G_{\theta} W_{\theta} ( \hat \theta_0 - \theta)  $$
where $F(\theta)$ represents the normal tendency terms due to physics, advection, etc., $G_{\theta}$ is a time-scale controlling the nudging strength, and $W_{\theta}$ is an additional weight in time or space to limit the nudging as described more below, while $\hat \theta_0$ is the time- and space-interpolated analysis field value towards which the nudging relaxes the solution.

Several options are available to control the nudging.

{\it a)} Nudging end-time and ramping. Nudging can be turned off during the simulation, as in dynamical initialization. Since turning nudging off suddenly can lead to noise, there is a capability for ramping the nudging down over a period, typically 1-2 hours to reduce the shock.

{\it b)} Strength of nudging. The timescale for nudging can be controlled individually for winds, temperature and moisture. Typically the namelist value of  0.0003 s-1 is used, corresponding to a timescale of about 1 hour, but this may be reduced for moisture where there may be less confidence in the analysis versus the details in the model.

{\it c)} Nudging in the boundary layer. Sometimes, since the analysis does not resolve the diurnal cycle, it is better not to nudge in the boundary layer to let the model PBL evolve properly, particularly the temperature and moisture fields. Each variable can therefore be selectively not nudged in the model boundary layer, the depth of which is given by the PBL physics.

{\it d)} Nudging at low levels. Alternatively the nudging can be deactivated for any of the variables below a certain layer throughout the simulation. For example, the lowest ten layers can be free of the nudging term. 

{\it e)} Nudging and nesting. Each of these controls is independently set for each domain when nesting, except for the ramping function, which has one switch for all domains.

\section{Surface Analysis Nudging}

The method used in surface analysis nudging is similar to that for 3-dimensional analysis nudging, except it only applies to the variables in the boundary layer. It utilizes a surface analysis file produced by an auxiliary program which may have higher temporal frequency than the 3-dimensional atmospherice analysis. 

\section{Flux-Adjusting Surface Analysis Nudging}

This is a surface analysis nudging method that not only nudges temperature and water mixing ratio at the first model level, but it also corrects soil temperature and moisture based on the differences in temperature and water vapor mixing ratio at the first model level (also refered to as Flux-Adjusting Surface Data Assimilation System or FASDAS). The differences of temperature and water vapor mixing ratio at the first model level are converted to surface flux adjustments that are applied to the prognostic equations of soil temperature and soil moisture. Through this indirect nudging, the surface temperature and water vapor mixing ratio from the model will become closer to the analysis. This method was developed for MM5 by \citet{alapaty08}.

\section{Observational or Station Nudging}

The observation-nudging capability allows the model to effectively assimilate temperature, wind and moisture observations from all platforms, measured at any location within the model domains and any time within a given data assimilation periods. With the observation-nudging formulation, each observation directly interacts with the model equations and thus the scheme yields dynamically and diabatically initialized analyses to support the applications that need regional 4-D full-field weather and/or to start regional NWP with spun-up initial conditions. The observation-nudging scheme, which is an enhanced version of the standard MM5 observation-nudging scheme, was originally implemented into ARW in Version 2.2 with additional enhancements applied in later versions. More details of the methods can be found in \citet{liu08} and \citet{deng09}, though not all of the modifications are included in the standard release.

Similar to analysis nudging, observation nudging is applied via an additional tendency term:

$$ {\partial a \mu_d \over \partial t}\left( x,y,z,t \right) = F_a\left( x,y,z,t \right) + \mu_d G_a { \sum\limits_{i=1}^{N} W^2_a\left(i,x,y,z,t)\right)\left[a_o\left( i \right) - a_m\left( x_i,y_i,z_i,t\right)\right] \over \sum\limits_{i=1}^{N} W_a\left(i,x,y,z,t)\right)} $$

\noindent
where $a$ is the quantity being nudged (which can be water vapor mixing ratio, potential temperature, u wind component, or v wind component), $\mu_d$ is the mass of the dry air in the column, $F_a$ represents the physical tendency terms for $a$, $G_a$ is the nudging strength for $a$, $N$ is the total number of observations, $i$ is the index to the current observation, $W_a$ is the spatiotemporal weighting function based on the temporal and spatial separation between the observation (located at $(x_i, y_i, z_i, t_i)$) and the current model location $(x, y, z, t)$, $a_o$ is the observed value of $a$, and $a_m(x_i,y_i,z_i,t)$ is the model value of $a$ interpolated to the observation location.  The innovation ($a_o-a_m$) for a given observation is calculated once at each time the nudging tendency term is updated and that innovation is then spread spatially. The nudging strength $G_a$ is the inverse of the e-folding time of the innovation (assuming that the physical tendency terms [$F_a$] were all zero and thus the observation nudging was the only factor modifying the variable $a$).  Note that while ARW reads in temperature observations it converts them to potential temperature for application in observation nudging.  Also, surface temperature and surface wind observations are adjusted to the lowest prognostic level for calculation of the innovation.

The user specifies the time period within the model integration over which observation nudging will be applied.  The user also specifies whether the strength at which the nudging is applied ramps down linearly with time at the end of the observation nudging period.  Within the time period the user specifies for application of observation nudging, individual observations are applied over a time window of user-specified length centered on the time of the observation.  Full weighting is applied over the half of the window centered at the observation time, and weighting ramps linearly with time in the quarters of the time window on either end of the time window.  A different time-length can be specified for surface observations than other observations.

Vertically, innovations from surface observations are spread based on user settings and can be dependent on the depth of the PBL and the current PBL regime.  Innovations from multi-level observations are vertically interpolated in pressure to model levels.  For single-level above-surface observations the innovation is applied from 75 hPa below the observation to 75 hPa above the observation with weights linearly decreasing with the difference in pressure between the observation and the level at which the innovation is being applied.  Single-level above-surface observations above (below) the PBL will not be spread below (above) the PBL since errors within the PBL may not be well-correlated with errors above the PBL.  The user can also choose to prevent any nudging of any observations from being applied within the PBL.

Horizontally, innovations from surface observations are spread along the surface, whereas innovations from above-surface observations are spread in pressure space (the vertical spreading described above is applied wherever the innovations are horizontally spread). The user specifies the horizontal radius of influence for above-surface observations and specifies a factor that this will be multiplied by to determine the radius of influence for surface observations.  The weighting of the innovation decreases non-linearly with distance from the observation within the radius of influence. The horizontal radius of influence for above-surface observations increases with decreasing pressure up until reaching double the specified radius of influence at 500 hPa.   An innovation from a single-level above-surface observation above (below) the PBL will not be spread horizontally to locations where it would be below (above) the PBL since errors within the PBL may not be well-correlated with errors above the PBL.  Since innovations calculated in a valley may not be representative of the error on a mountain top, the weighting for surface observations is decreased based on the difference between the surface pressure at the location of the observation and at the location the innovation is being applied.

More details regarding observation nudging in ARW can be found in a brief guide on this topic (http://www2.mmm.ucar.edu/wrf/users/docs/ObsNudgingGuide.pdf). 

 
\section{Spectral Nudging}

Spectral nudging is another way to nudge the model solution to either analysis or any forcing data. The key difference between this method and that of grid- or obs-nudging is that it is more selective in the scales one would like to nudge. In contrast to directly nudge model variables toward analysis or observation, this method decomposes the difference fields spectrally, and nudges toward the longer waves that correspond to the analysis. The spectral coefficients from the selected waves are then transformed back from the wave space to physical space and added to the model as tendencies. This technique was originally implemented by \citet{miguez04} to the Reginoal Atmospheric Modeling System, and later to ARW in 2010.

Spectral nudging uses the same input data as for grid-nudging, and nudges model variables u, v, temperature, and geopotential height. Water mixing ratio nudging was added in Version 4.0. Applications described above for grid-nudging can be considered using spectral nudging too.

Options to control the grid-nudging are also used for spectral nudging (such as nudging end time and ramping, nudging strength, controls of nudging in the boundary layer or lower levels and so on). In addition, there is an option to control the ramping in the vertical between the layer without nudging to the layer where the nudging is on. The number of waves to nudge in the x and y directions are user-defined.

