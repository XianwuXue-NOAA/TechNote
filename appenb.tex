\chapter{List of Symbols}
\label{List of Symbols}

Symbols used in this document are listed in alphabatical order in this appendix.
\\[2ex]

\begin{tabular}{ l p{5.5in} }

{\em Symbols}  & {\em Definition} \\   
\\
$a$            & generic variable \\ 
$A$            & coefficient (Chapter \ref{filter_chap}), base-state lapse rate constant (Chapter \ref{initialization_chap})  \\  
{\bf B}        & background error covariance matrix \\  
$c$            & scalar coefficient \\  
$c_s$          & speed of sound      \\  
$C_k$          & a constant used in TKE closure \\  
$Cr$           & Courant number      \\  
$Cr_{max}$     & maximum Courant number      \\  
$Cr_{theory}$  & Courant number from Table 3.1 \\  
$Cr_{\beta}$   & activation Courant number in vertical velocity damping \\  
$C_s$          & a constant used in eddy viscosity calculation \\  
$D$            & deformation \\
$D_{nm}$       & deformation tensor, where $n, m$ = 1, 2 and 3 \\  
$e$            & cosine component of the Coriolis term (Chapters \ref{equation_chap}, \ref{discretization_chap}); turbulent kinetic energy (Chapter \ref{filter_chap}) \\
{\bf E}        & observation error covariance matrix \\
$f$            & sine component of the Coriolis term \\
$F$            & forcing terms for $U$, $V$, $W$, $\Theta$ and $Q_m$  \\
{\bf F}        & representivity error covariance matrix \\
$F_{X_{cor}}$  & Coriolis forcing terms for $X = $ $U$, $V$, and $W$  \\
$F_{1,2}$      & coefficients for weighting functions in specified boundary condition \\
$g$            & acceleration due to gravity \\
$G_k$          & regression coefficient \\
$H$            & observation operator \\
$J$            & cost function \\
$K_{dh,dv}$    & horizontal and vertical eddy viscosity for gravity wave absorbing layer \\
$K_{h,v}$      & horizontal and vertical eddy viscosities \\
\end{tabular}

\newpage
\vskip 5pt
\begin{tabular}{ l p{5.5in} }

{\em Symbols}  & {\em Definition} \\
\\
$l_0$          & minimum length scale for dissipation \\
$l_{h,v}$      & horizontal and vertical length scales for turbulence \\
$l_{cr}$       & critical length scale for dissipation \\
$L$            & latent heat of condensation \\
$L_{x,y}$      & periodicity length in $x$ and $y$ \\
$m$            & map scale factor \\ 
$n_s$          & ratio of the RK3 time step to the acoustic time step \\
$N$            & Brunt-V\"ais\"al\"a frequency \\
$p$            & pressure \\
$p'$           & perturbation pressure \\
$p_0$          & reference sea-level pressure \\
$p_h$          & hydrostatic pressure \\
$p_{ht,hs}$    & hydrostatic pressure at the top and surface of the model \\
$p_{dht,dhs}$  & dry hydrostatic pressure at the top and surface of the model \\
$p_s$          & surface pressure \\
$P_r$          & Prandtl number \\
$q$            & generic scalar \\
$q_{c,i,r,s}$  & mixing ratios for cloud water, ice, rain water and snow \\
$q_m$          & generic mixing ratios for moisture \\
$q_v$          & mixing ratio for water vapor \\
$q_{vs}$       & saturation mixing ratio for water vapor \\
$Q_m$          & generic coupled moisture variable \\
$r$            & relative humidity   \\
$r_e$          & radius of earth \\
$R$            & remaining terms in equations  \\
$R_d$          & gas constant for dry air \\
$R_v$          & gas constant for water vapor \\
$t$            & time \\
$\Delta t$     & a full model time step \\
$T$            & temperature \\
$T_0$          & reference sea-level temperature \\
$u$            & horizontal component of velocity in $x$-direction \\
$U$            & coupled horizontal component of velocity in $x$-direction (Chapters \ref{equation_chap}, \ref{discretization_chap}, \ref{lbc_chap}, \ref{nesting_chap}); control variable transform (Chapter \ref{var_chap}) \\
$U_h$          & horizontal correlation \\
$U_p$          & multivariate covariance \\
$U_v$          & vertical covariance \\
$v$            & horizontal component of velocity in $y$-direction \\
$\bf v$        & three dimensional vector velocity \\
$V$            & coupled horizontal component of velocity in $y$-direction \\
$\bf V$        & three dimensional coupled vector velocity \\
$w$            & vertical component of velocity \\
$W$            & coupled vertical component of velocity \\

\end{tabular}

\newpage
\begin{tabular}{ l p{5.5in} }
{\em Symbols}  & {\em Definition} \\
\\
$W_k$          & regression coefficient \\
$z$            & height \\
$z_d$          & depth of damping layer \\
$z_{top}$      & height of model top \\

$\alpha$       & inverse density of air \\  
$\alpha'$      & perturbation inverse density of air \\  
$\bar\alpha$   & inverse density of air for the reference state \\  
$\alpha_d$     & inverse density of dry air \\  
$\alpha_r$     & local rotation angle between $y$-axis and the meridian \\  
$\beta$        & off-centering coefficient for semi-implicit acoustic step \\  
$\gamma$       & ratio of heat capacities for dry air at constant pressure and volume \\
$\gamma_d$     & divergence damping coefficient \\
$\gamma_e$     & external mode damping coefficient \\
$\gamma_g$     & damping coefficient for upper boundary gravity wave absorbing layer \\
$\gamma_r$     & Rayleigh damping coefficient \\
$\epsilon$     & molecular weight of water over the molecular weight of dry air (Chapter \ref{filter_chap}); true background error (Chapter \ref{var_chap}) \\

$\eta$         & terrain-following hydrostatic-pressure vertical coordinate \\
$\dot\eta$     & contravariant `vertical' velocity or coordinate velocity \\
$\theta$       & potential temperature  \\
$\theta_e$     & equivalent potential temperature  \\
$\theta_m$     & moist potential temperature  \\
$\Theta$       & coupled potential temperature  \\  
$\mu$          & hydrostatic pressure difference between surface and top of the model  \\  
$\bar\mu$      & reference state hydrostatic pressure difference between surface and top of the model  \\  
$\mu_d$        & dry hydrostatic pressure difference between surface and top of the model  \\  
$\tau$         & acoustic time (Chapter \ref{discretization_chap}), vertical structure function for Rayleigh damping (Chapter \ref{filter_chap}) \\  
$\tau_{nm}$    & stress tensor (Chapter \ref{filter_chap}) where $n. m$ = 1, 2 and 3 \\  
$\Delta \tau$  & acoustic time step \\  
$\phi$         & geopotential (Chapters \ref{equation_chap}, \ref{discretization_chap}, \ref{initialization_chap}); latitude (Chapter \ref{var_chap}) \\  
$\bar \phi$    & geopotential for reference state \\  
$\phi'$        & perturbation geopotential \\  
$\Phi$         & generic prognostic variable (coupled) \\  
$\psi$         & generic variable (Chapter \ref{lbc_chap}) \\  
$\psi'$        & streamfunction increment \\  
$\chi'$        & velocity potential increment \\  
$\omega$       & same as $\dot\eta$ \\  
$\Omega$       & coupled coordinate velocity \\  
$\Omega_e$     & angular rotation rate of the earth \\  
\\
\end{tabular}

\newpage
\normalsize
\begin{tabular}{ l p{5.5in} }
{\em Subscripts/Superscripts}  & {\em Definition} \\
\\
$()_d$         & dry \\
$()_h$         & hydrostatic \\
$()_0$         & base state sea-level constant \\
$\overline{()}$ & reference state \\
$()'$          & perturbation from reference state \\
$()^{t^*}$     & full value at a Runge-Kutta step \\
$()''$         & perturbation from Runge-Kutta step value in acoustic steps \\
\end{tabular}

